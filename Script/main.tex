\documentclass[12pt,a4paper]{article}
\usepackage[utf8]{inputenc}
\usepackage[francais]{babel}
\usepackage{amsmath}
\usepackage{amsfonts}
\usepackage{amssymb}
\usepackage{stmaryrd}
\usepackage{float}
\usepackage{systeme}

\usepackage{tikz}
\setlength{\oddsidemargin}{-18pt}
\setlength{\evensidemargin}{-18pt}
\setlength{\textwidth}{481pt}
\setlength{\textheight}{710pt}
\setlength{\headheight}{-53pt}
\setlength{\footskip}{18pt}
\usepackage[hyperindex]{hyperref}
\newtheorem{dfn}{\textbf{Définition}}[subsection]
\newtheorem{thm}[dfn]{\textbf{Théoreme}}
\newtheorem{prop}[dfn]{\textbf{Proposition}}
\newtheorem{cor}[dfn]{\textbf{Corollaire}}
\newtheorem{lem}[dfn]{\textbf{Lemme}}
\usepackage[nothm]{thmbox}

\usepackage{graphicx}
\usepackage{caption}

\newcommand{\rot}{\text{rot}}


\usepackage[
   backend=biber,
   sorting=nyt,
   format=plain]{biblatex}
\bibliography{sample.bib}

\numberwithin{equation}{section}
\def\proof{\begin{thmbox}[M]{\textbf{Preuve :}}}
\def\endproof{\end{thmbox}}
\begin{document}
\title{Analyse spectrale d'écoulements fluides et modèles d'amortissement}
\author{Lesage Adrien}
\maketitle
\newpage
\tableofcontents
\newpage
\section{Introduction aux équations d'Euler à surface libre}
\subsection{Hypothèses}

Tout le long de ce mémoire, nous nous intéresserons à l'étude d'un fluide dont nous supposerons les hypothèses suivantes vérifiées
\begin{list}{}{}
    \item[$(\textbf{H}_1)$ : ] Le fluide est \textit{homogène} et \textit{incompressible}.
    \item[$(\textbf{H}_2)$ : ] Le fluide est \textit{non visqueux}.
    \item[$(\textbf{H}_3)$ : ] Le fluide est \textit{irrotationnel}.
\end{list}

Soit $\textbf{X} = \mathbb{R}^d$ avec $d \in \{1,2\}$. On souhaite étudier un tel fluide évoluant dans le domaine $$\Omega_t = \left\{ (x,z) \in \textbf{X}\times\mathbb{R}  \,|\quad b(x) - H_0 \leq z \leq \zeta(t,x) \,\right\}$$
\begin{list}{}{}
\item[\textbullet] $b : \textbf{X}\rightarrow \mathbb{R}$ désigne les variations du fond du fluide, de profondeur caractéristique $H_0 > 0$
\item[\textbullet] le graphe $\zeta : \mathbb{R}_+\times \textbf{X}$ désigne la surface du fluide, de sorte que les hypothèses suivante soient vérifiées
\end{list}


\begin{list}{}{}
    \item[$(\textbf{H}_4)$ : ] Les particules de fluide ne traversent pas le fond.
    \item[$(\textbf{H}_5)$ : ] Les particules de fluide ne traversent pas la surface.
    \item[$(\textbf{H}_6)$ : ]  En tout point $x\in\textbf{X}$ et en tout instant $t\geq 0$, l'épaisseur du fluide $\zeta(t,x) - (b(x)-H_0)$ est supérieure à une constante $H_0>0$ indépendante de $x$ et de $t$.
    \item[$(\textbf{H}_7)$ : ] La pression  du fluide à la surface est égale à la pression atmosphérique $P_{\text{atm}}$ (on néglige les tensions de surface).
    \item[$(\textbf{H}_8)$ : ] Le fluide est soumis à une force de gravité d'intensité $g$ et de direction opposée au vecteur $\textbf{z} = (0_{\textbf{X}},1)$.
    \item[$(\textbf{H}_9)$ : ] Le fluide est au repos à l'infini et $\lim\limits_{\|x\|\rightarrow \infty}\zeta(x,t) = 0$
\end{list}

\begin{figure}[H]
    \centering
    \begin{tikzpicture}[scale=1.5]
        % Axes
        \draw[->] (-0.2,0) -- (5.5,0) node[right] {$x$};
        \draw[->] (0,-1.3) -- (0,0.5) node[above] {$z$};
        
        % Courbes de surface
        \draw[thick, domain=0.4:5, smooth, variable=\x, blue] plot ({\x},{0.05*sin(3*\x +1 r)-0.1*sin(2*\x r)}) node[above] {$\zeta$};
        \draw[thick,dashed, domain=0.1:5.4, smooth, variable=\x, blue] plot ({\x},{0.05*sin(3*\x +1 r)-0.1*sin(2*\x r)}) ;
        
        \draw[thick, domain=0.4:5, smooth, variable=\x, red] plot ({\x},{-1 - 0.05*sin(\x +1 r) - 0.1*sin(2*\x r)}) node[above] {$b - H_0$};
        \draw[thick, domain=0.1:5.4, dashed, smooth, variable=\x, red] plot ({\x},{-1 - 0.05*sin(\x +1 r) - 0.1*sin(2*\x r)});

        \node at (-0.4,0) {$0$};
        \node at (-0.4,-1) {$-H_0$};
        \node at (2.15,-0.5) {$\Omega_t$};
    \end{tikzpicture}
    \caption{Représentation des équations d'Euler à surface libre}
    \label{fig:surface}
\end{figure}

\subsection{Formulation des équations d'Euler}
\subsubsection{Formulation de l'évolution du champ de vitesse}
Pour commencer, nous allons adopter un point de vue lagrangien:
\begin{list}{}{}
    \item[\textbullet] Posons $\left(\begin{array}{l}
         \mathcal{X}\\
         \mathcal{Z}
    \end{array}\right) :  \mathbb{R}_+ \rightarrow \textbf{X}\times\mathbb{R}$ la fonction qui suit le déplacement d'une particule de fluide de position initiale $\left(\begin{array}{l}
         \mathcal{X}\\
         \mathcal{Z}
    \end{array}\right)(0)  \in \Omega_0$.
    \item[\textbullet] Posons $\textbf{U}(t,x,z)$ la vitesse de la particule de fluide située en $(x,z) \in \Omega_t$ à l'instant $t$.
\end{list}
Les hypothèses $(\textbf{H}_4)$ et $(\textbf{H}_5)$ nous donnent que $\left(\begin{array}{l}
         \mathcal{X}\\
         \mathcal{Z}
\end{array}\right)\in \Omega_t$ pour tout instant $t \geq 0$. On a alors la relation
    
\begin{equation}
\frac{d}{dt} \left(\begin{array}{l}
         \mathcal{X}\\
         \mathcal{Z}
\end{array}\right) = \textbf{U}(t,\mathcal{X}(t), \mathcal{Z}(t)) 
\end{equation}

En dérivant une fois de plus, on obtient 
$$
\frac{d^2 }{dt^2}\left(\begin{array}{l}
         \mathcal{X}\\
         \mathcal{Z}
\end{array}\right)  = \frac{\partial \textbf{U}(t,\mathcal{X}(t),\mathcal{Z}(t))}{\partial t} + \text{Jac}(\textbf{U})\frac{d}{d t}\left(\begin{array}{l}
         \mathcal{X}\\
         \mathcal{Z}
\end{array}\right) 
$$
    Ou, autrement dit,
    
$$
\frac{d^2 }{dt^2}\left(\begin{array}{l}
         \mathcal{X}\\
         \mathcal{Z}
\end{array}\right)= \left(\frac{\partial\textbf{U}}{\partial t}+ (\textbf{U}\cdot \nabla ) \textbf{U}\right)(t,\mathcal{X}(t),\mathcal{Z}(t)) 
$$

L' hypothèse $(\textbf{H}_2)$ nous dit que les particules ne s'influencent pas mutuellement en dehors de la pression. Chaque particule n'est alors soumis qu'à deux forces: la force de gravité $ F_\text{g} = -\rho g\textbf{z}$ due à l'hypothèse $(\textbf{H}_8)$ et la force de pression $F_\text{P} = - \nabla P$ ou la fonction $P$ désigne la pression du fluide. La deuxième loi de Newton donne alors

$$
\rho\frac{d^2 }{dt^2}\left(\begin{array}{l}
         \mathcal{X}\\
         \mathcal{Z}
\end{array}\right)(t) = - \rho g \textbf{z} -\nabla P(t,X(t),\mathcal{Z}(t))
$$
Ce qui nous donne les équations d'Euler définies par le système suivant.

\begin{equation} 
\tag{$\textbf{E}_1$} \label{eq_E1}
\rho \left( \frac{\partial \textbf{U}}{\partial t} + (\textbf{U}\cdot \nabla ) \textbf{U} \right) + \rho g\textbf{z} + \nabla P = 0
\end{equation}


\subsubsection{Formulation de la condition d'incompressibilité}
L'hypothèse $(\textbf{H}_1)$ se traduit par
$$\rho(t,x,z) = \rho_0$$
En conséquence, la loi de conservation de la masse d'un fluide
\begin{equation} \label{mass_conservation}
    \frac{\partial \rho}{\partial t} +\text{div}(\rho\textbf{U}) = 0
\end{equation}
devient
\begin{equation} \tag{$\textbf{E}_2$} \label{divergence_free}
    \text{div} (\textbf{U}) = 0
\end{equation}

\subsubsection{Formulation des conditions aux bords et de l'évolution de la surface}

L'hypothèse $(\textbf{H}_4)$ s'exprime ainsi

\begin{equation}
    \tag{${\textbf{E}}_3$} \label{eq_E2}
    \textbf{U}.\textbf{n} = 0 \text{  sur }\{z = b(x)-H_0\}
\end{equation} 

où \textbf{n} désigne le vecteur normal extérieur au domaine $\Omega_t$.\\

Cette formule est une réécriture immédiate de $(\textbf{H}_4)$  car $b-H_0$ ne dépend pas du temps. En revanche, comme $\zeta$ dépend du temps, la caractérisation de l'hypothèse $(\textbf{H}_5)$ est moins immédiate et est l'objet de la proposition suivante.



\begin{prop} L'hypothèse $(\textbf{H}_5)$ nous donne que
    \begin{equation} \tag{$\textbf{E}_4$} \label{eq_E3}
        \frac{\partial \zeta}{\partial t}(t,x) - \sqrt{1+ \| \nabla_\textbf{X}\zeta\|^2} \textbf{U} \cdot \textbf{n} = 0 \quad\text{dans}\,\{z = \zeta(t,x)\}
    \end{equation}
    où $\textbf{n}$ désigne le vecteur normal extérieur au domaine $\Omega_t$.
\end{prop}
\begin{proof}
\textit{Etape 1 : }Commençons par déterminer $\textbf{n}$ en fonction de $t$, $x$, et $\zeta$. A $t$ fixé, la surface est paramétrée par l'application $$S_t: x\in \textbf{X} \mapsto \left(\begin{array}{l}
         x\\
         \zeta(t,x)
    \end{array}\right).$$
Sa différentielle en $x$ est 
$$DS_t(x) : v \in \textbf{X} \mapsto DS_t(x).v = \left(\begin{array}{l}
         v\\
         \nabla_X\zeta(t,x) \cdot v
    \end{array}\right).$$
Le vecteur normal \textbf{n} vérifie alors $$\textbf{n}\cdot (DS_t(x).v) = 0$$
pour tout $v \in \textbf{X}$. En posant $\textbf{n} = \left(\begin{array}{l}
         n_1\\
         n_2
    \end{array}\right)$, on trouve alors
    $$ \left(n_1 + n_2\nabla_X\zeta(t,x)\right)\cdot v = 0$$
pour tout $v\in\textbf{X}$. On a alors
$$ n_1 =  - n_2\nabla_X\zeta(t,x)$$
Donc $\textbf{n} = n_2\left(\begin{array}{l}
         - \nabla_X\zeta(t,x)\\
         1
    \end{array}\right)$
Pour que la normale soit dirigée vers l'exterieur du domaine $\Omega_t$, il faut $n_2>0$. Comme $\|\textbf{n}\| = 1$, on trouve finalement \begin{equation} \label{normal_surface_vector}
\textbf{n} = \frac{1}{\sqrt{1+\|\nabla_X\zeta(t,x)\|^2}}\left(\begin{array}{l}
         - \nabla_X\zeta(t,x)\\
         1
    \end{array}\right).\end{equation}
\textit{Etape 2 : }Considérons à nouveau $\left(\begin{array}{l}
         \mathcal{X}\\
         \mathcal{Z}
\end{array}\right)$ 
la trajectoire d'une particule et supposons qu'elle est initialement située à la surface. L'hypothèse nous donne qu'alors la particule ne peut traverser la surface ni dans un sens, ni dans l'autre. Il y a alors, pour tout temps $t \geq 0$, l'égalité suivante:
$$\mathcal{Z} - \zeta( . ,\mathcal{X}) = 0 .$$
En dérivant cette égalité, il vient
\begin{align*} 
    0 &= \frac{d\mathcal{Z}}{dt} - \frac{\partial \zeta}{\partial t}(.,\mathcal{X}) - \nabla_\textbf{X}\zeta(.,\mathcal{X}) \cdot \frac{d \mathcal{X}}{dt} \\ 
    &=  \textbf{U}\cdot  \left(\begin{array}{l}
         - \nabla_X\zeta(t,x)\\
         1
    \end{array}\right) - \frac{\partial \zeta}{\partial t}(.,\mathcal{X}) 
    \\&= \sqrt{1 + \|\zeta\|^2}\textbf{U}  \cdot \textbf{n} - \frac{\partial \zeta}{\partial t}(.,\mathcal{X}) 
\end{align*}



\end{proof}

\paragraph{Pour résumer,} en combinant (\ref{eq_E1}), (\ref{divergence_free}), (\ref{eq_E2}) et (\ref{eq_E3}), on obtient les équations d'Euler à surface libre
\begin{equation} \tag{\textbf{E}}
    \left\{ 
    \begin{split}
        &\rho \left ( \frac{\partial \textbf{U}}{\partial t} + (\textbf{U}\cdot \nabla ) \textbf{U}+  g\textbf{z}  \right) + \nabla \textbf{P} = 0 &\, \\
        &\text{div}(\textbf{U}) = 0 &\, \\
        &\textbf{U}.\textbf{n} = 0 &~~\text{dans}\,\{z = b(x)-H_0\}\\
        &\frac{\partial \zeta}{\partial t}(t,x) - \sqrt{1+ \| \nabla_\textbf{X}\zeta\|^2} \textbf{U} \cdot \textbf{n} = 0 &\text{dans}\,\{z = \zeta(t,x)\} 
    \end{split}
    \right.
\end{equation}
\paragraph{Remarque :} L'égalité vectorielle suivante
$$(\textbf{U}\cdot \nabla ) \textbf{U} = \rot(\textbf{U})\wedge \textbf{U} + \frac{1}{2}\nabla |\textbf{U}|^2$$
Nous permet de réécrire (\ref{eq_E1}) sous la forme
\begin{equation} \label{eq_E1bis}
    \rho \left( \frac{\partial \textbf{U}}{\partial t} +\rot(\textbf{U})\wedge \textbf{U} + \frac{1}{2}\nabla |\textbf{U}|^2+  g\textbf{z}  \right) + \nabla \textbf{P} = 0 
\end{equation}

Ainsi, l'hypothèse $(\textbf{H}_3)$, qui se traduit par $\rot(\textbf{U})=0$, permet alors d'avoir 
\begin{equation} \tag{$\textbf{E}_{1,\text{bis}}$} \label{eq_E1bis}
        \rho \left( \frac{\partial \textbf{U}}{\partial t} +\frac{1}{2}\nabla  |\textbf{U}|^2 +  g\textbf{z}\right) + \nabla \textbf{P} = 0 
\end{equation}

\subsection{Formulation de Bernoulli}

Grâce notamment à $(\textbf{H}_6)$, on sait que $\Omega_t$ est simplement connexe à tout instant $t$, et comme $\rot(\textbf{U}) = 0$ par l'hypothèse $(\textbf{H}_3)$, on en déduit l'existence d'une fonction $\Phi(t,.,.): \textbf{X}\times \mathbb{R} \rightarrow \mathbb{R}$ telle que $$\textbf{U} = \nabla \phi  \;.$$    

Ceci permet de réécrire le système (\ref{eq_E1bis}):

\begin{align*}
    0 = &\rho \left ( \frac{\partial \nabla \phi }{\partial t} +\frac{1}{2}\nabla  |\nabla \phi|^2  +  g\textbf{z} \right)+ \nabla P 
    \\
    =&\nabla \left ( \rho \left ( \frac{\partial \phi }{\partial t} +\frac{1}{2}|\nabla \phi|^2 +  gz \right) + P \right)
\end{align*}

Une condition suffisante pour vérifier (\ref{eq_E1bis}) est alors
\begin{equation*} 
  \frac{\partial \phi }{\partial t} +\frac{1}{2}|\nabla \phi|^2 +  gz   = \frac{1}{\rho}\left( - P\right)
\end{equation*}



De plus, comme $\phi$ n'est définie qu'à une constante indépendante de l'espace près et comme seules les dérivées de $\phi$ par rapport aux variables d'espace interviennent dans les équations, on peut aussi poser
\begin{equation} \tag{$\textbf{B}_1$} \label{eq_B1}
  \frac{\partial \phi }{\partial t} +\frac{1}{2}|\nabla \phi|^2 +  gz   = \frac{1}{\rho}\left(P_\text{atm} - P\right)
\end{equation}
Où la constante $P_\text{atm}$ désigne la pression de l'air ambiant, supposée constante. De cette façon, par l'hypothèse $(\textbf{H}_7)$, la pression disparaît de cette équation sur la surface $\{z = \zeta(t,x)\}$.
Et comme $\text{div}(\nabla \phi) = \Delta \phi$, l'équation (\ref{divergence_free}) devient alors

\begin{equation} \tag{$\textbf{B}_2$} \label{eq_B2}
    \Delta \phi = 0.
\end{equation}

L'équation (\ref{eq_E2}), devient

\begin{equation} \tag{$\textbf{B}_3$} \label{eq_B3}
    \frac{\partial \phi}{\partial \textbf{n}} = 0 ~~~ \text{dans}\,\{z = b(x)-H_0\} .
\end{equation}

L'équation (\ref{eq_E3}), devient

\begin{equation} \tag{$\textbf{B}_4$} \label{eq_B4}
\frac{\partial \zeta}{\partial t}  - \sqrt{1+ \| \nabla_\textbf{X}\zeta\|^2}\frac{\partial \phi}{\partial\textbf{n}} = 0~~~\text{dans}\,\{z = \zeta(t,x)\} .
\end{equation}

Sachant (\ref{normal_surface_vector}), on connaît et on utilisera souvent l'équivalence entre (\ref{eq_B4}) et l'équation suivante


\begin{equation} \label{eq_B4_bis}
    \frac{\partial \zeta}{\partial t}  + \nabla_\textbf{X}\zeta\cdot\nabla_\textbf{X}\phi - \frac{\partial\phi}{\partial z} = 0 ~~~\text{dans }\{z = \zeta(t,x)\} .
\end{equation}

De manière analogue, on a aussi équivalence entre (\ref{eq_B3}) et 

\begin{equation} \label{eq_B3_bis}
    \nabla_\textbf{X}b\cdot\nabla_\textbf{X}\phi - \frac{\partial\phi}{\partial z} = 0 ~~~\text{dans }\{z = b(x)\} .
\end{equation}


\subsection{Adimensionnement des paramètres du système}

\begin{figure}[H]
    \centering
    \begin{tikzpicture}[scale=1.5]
        % Axes
        \draw[->] (-0.2,0) -- (5.5,0) node[right] {$x$};
        \draw[->] (0,-2.8) -- (0,0.5) node[above] {$z$};

        \draw[ gray] (-0.1,-2.5) -- (5.5,-2.5);
        % Courbes de surface
        \draw[very thick, domain=0.4:5, smooth, variable=\x, blue] plot ({\x},{0.2*sin(3*\x +1 r)-0.4*sin(2*\x r)}) node[above] {$\zeta$};
        \draw[very thick,dashed, domain=0.1:5.4, smooth, variable=\x, blue] plot ({\x},{0.2*sin(3*\x +1 r)-0.4*sin(2*\x r)}) ;
        
        \draw[very thick, domain=0.4:5, smooth, variable=\x, red] plot ({\x},{-2.5 - 0.2*sin(\x +1 r) - 0.4*sin(2*\x r)}) node[above] {$b - H_0$};
        \draw[very thick, domain=0.1:5.4, dashed, smooth, variable=\x, red] plot ({\x},{-2.5 - 0.2*sin(\x +1 r) - 0.4*sin(2*\x r)});

        \node at (-0.4,0) {$0$};
        \node at (-0.4,-2.5) {$-H_0$};

        \draw[->, dashed, gray] (2.3,0) -- (2.3,{0.2*sin(3*2.3 +1 r)-0.4*sin(2*2.3 r)}) node[midway, right] {$a_\text{surf}$};
        \draw[->, dashed, gray] (3.9,-2.5) -- (3.9,{-2.5 - 0.2*sin(3.9 +1 r) - 0.4*sin(2*3.9 r)}) node[midway, right] {$a_\text{bot}$};
        \draw[->, dashed, gray] (1,-0.22) -- (3.8,-0.22) node[below, midway] {$L_1$};
    \end{tikzpicture}
    \caption{Représentation des grandeurs caractéristiques pour $\textbf{X} = \mathbb{R}$ }
    \label{fig:surface}
\end{figure}
Tout comme dans \cite{MT} et  \cite{Lannes}, on va commencer par réécrire ces équations en faisant apparaître les rapports entre les différentes grandeurs caractéristiques qui entrent en jeu. Ainsi la négligeabilité d'un paramètre relativement à un autre s'impactera sur nos équations sous la forme d'une possible simplification de un ou plusieurs termes. Une équation ne faisant intervenir que des rapports adimmensionés est dites adimmensionée (ou adimensionnelle). 
\\

Dans notre cas, les grandeurs caractéristiques de notre système sont

\begin{list}{\textbullet}{}
    \item $H_0$ est la profondeur du système.
    \item $a_{\text{surf}} = max(|\zeta|)$ est l'amplitude de la surface du fluide.
    \item $a_{\text{bot}} = max(|b|)$ est l'amplitude du fond du fluide.
    \item $L_1$ est la longueur d'onde de $\zeta$ dans la direction $\textbf{e}_1$.
    \item $L_2$ est la longueur d'onde de $\zeta$ dans la direction transversale $\textbf{e}_2$ (si $\textbf{X} = \mathbb{R}^2)$.
    
 \end{list}
\,
\\

On étudiera les rapports suivants:

\begin{list}{\textbullet}{}
    \item $\varepsilon = \frac{a_{\text{surf}}}{H_0}$ le coefficient de \textit{non-linéarité} du système.
    \item $\mu =\frac{H_0^2}{L_1^2}$ le coefficient de \textit{profondeur}.
    \item $\beta = \frac{a_{\text{bot}}}{H_0}$ est le coefficient de \textit{dénivellation} du fond.
    \item $\gamma = \frac{L_1}{L_2}$ est le coefficient de \textit{transversalité}.
    \item $\epsilon = \frac{ a_{\text{surf}}}{L_1}$ le coefficient d'\textit{amplitude}.
    \item $t_0 = \frac{L_1}{\sqrt{gH_0}}$ est l'échelle de temps caractéristique.
\end{list}

~\\
Pour adimensionner nos équations, on pose l'isomorphisme linéaire suivant:

  
\begin{equation}
\begin{aligned}
      \mathcal{I}:~~&\mathbb{R}\times\textbf{X}\times\mathbb{R} &\longrightarrow ~~~ &\textbf{X}\times \mathbb{R} & \\
      &(t,(x_1,x_2),z) &\longmapsto &\left(\frac{t}{t_0},\left(\frac{x_1}{L_1},\frac{x_2}{L_2} \right),\frac{z}{H_0}\right) ~~~&\text{si }\textbf{X}=\mathbb{R}^2\\
      &(t, x ,z) &\longmapsto &\left(\frac{t}{t_0},\frac{x}{L_1},\frac{z}{H_0}\right) ~~~&\text{si }   \textbf{X}=\mathbb{R}
\end{aligned}
\end{equation}

    
   
    


~\\

On effectue le changement de variable suivant:

\begin{list}{\textbullet}{}
    \item $(t',x',z') = \mathcal{I} (t,x,z)$
    \item $\zeta' = \dfrac{1}{a_{\text{surf}}}(\zeta \circ \mathcal{I}^{-1})$ 
    \item $b' = \dfrac{1}{a_{\text{bot}}}(b\circ\mathcal{I}^{-1})$
     \item $\phi' = \dfrac{1}{g t_0 a_{\text{surf}}}(\phi \circ \mathcal{I}^{-1})$
     \item $P' = \dfrac{t_0^2}{H_0^2\rho_0} (P \circ \mathcal{I}^{-1})~~~~~~~$   (et $P_{\text{atm}}' = \dfrac{t_0^2}{H_0\rho_0} P_\text{atm}$)
\end{list}

\begin{figure}[H]
    \centering
    \begin{tikzpicture}[scale=1.5]
        % Axes
        \draw[->] (-0.2,0) -- (5.5,0) node[right] {$x'$};
        \draw[->] (0,-2.8) -- (0,0.5) node[above] {$z'$};

        \draw[ gray] (-0.1,-2.5) -- (5.5,-2.5);
        % Courbes de surface
        \draw[very thick, domain=0.4:5, smooth, variable=\x, blue] plot ({\x},{0.2*sin(3*\x +1 r)-0.4*sin(2*\x r)}) node[above] {$\varepsilon\zeta'$};
        \draw[very thick,dashed, domain=0.1:5.4, smooth, variable=\x, blue] plot ({\x},{0.2*sin(3*\x +1 r)-0.4*sin(2*\x r)}) ;
        
        \draw[very thick, domain=0.4:5, smooth, variable=\x, red] plot ({\x},{-2.5 - 0.2*sin(\x +1 r) - 0.4*sin(2*\x r)}) node[above] {$\beta b' - 1$};
        \draw[very thick, domain=0.1:5.4, dashed, smooth, variable=\x, red] plot ({\x},{-2.5 - 0.2*sin(\x +1 r) - 0.4*sin(2*\x r)});

        \node at (-0.4,0) {$0$};
        \node at (-0.4,-2.5) {$-1$};

        \draw[->, dashed, gray] (2.3,0) -- (2.3,{0.2*sin(3*2.3 +1 r)-0.4*sin(2*2.3 r)}) node[midway, right] {$\mathcal{O}(\varepsilon)$};
        \draw[->, dashed, gray] (3.9,-2.5) -- (3.9,{-2.5 - 0.2*sin(3.9 +1 r) - 0.4*sin(2*3.9 r)}) node[midway, right] {$\mathcal{O}(\beta)$};
        \draw[->, dashed, gray] (1,-0.22) -- (3.8,-0.22) node[below, midway] {$\mathcal{O}(1)$};
    \end{tikzpicture}
    \caption{Représentation des courbes adimensionnées}
    \label{fig:surface}
\end{figure}
\subsubsection{Adimensionnement des équations de Bernoulli pour $\textbf{X} = \mathbb{R}$}
Pour $\textbf{X} = \mathbb{R}$, l'adimensionnement de (\ref{eq_B1}) s'effectue ainsi:
\paragraph{Adimensionnement de (\ref{eq_B1}):}
Après changement de variable, (\ref{eq_B1}) devient
\begin{equation*}
     \frac{g t_0 a_{\text{surf}}}{t_0}\partial_{t'} \phi' + \frac{1}{2}\left(\frac{(g t_0 a_{\text{surf}})^2}{L_1^2}(\partial_{x'}\phi')^2+ \frac{(g t_0 a_{\text{surf}})^2}{H_0^2}(\partial_{z'}\phi')^2\right) + g H_0 z' = \frac{1}{\rho_0}\frac{H_0^2\rho_0}{t_0^2}(P'-P'_{\text{atm}}).
\end{equation*}
Après factorisation par $ga_\text{surf}$, on trouve
\begin{equation*}
    \partial_{t'} \phi' + \frac{1}{2}\left(\frac{g t_0^2 a_{\text{surf}}}{L_1^2}(\partial_{x'}\phi')^2+ \frac{g t_0^2 a_{\text{surf}}}{H_0^2}(\partial_{z'}\phi')^2\right) + \frac{z'}{\varepsilon}  = \frac{H_0^2}{t_0^2ga_\text{surf}}(P'-P'_{\text{atm}}).
\end{equation*}
Remarquons alors que
\begin{equation}
\begin{split}
    g t_0^2 a_{\text{surf}} &= g\left(\frac{L_1}{\sqrt{gH_0}}\right)^2a_\text{surf}\\
    &=  \frac{L_1^2}{H_0}a_\text{surf}\\
    &= \frac{H_0}{\mu}a_\text{surf} \\
    &= H_0^2\frac{\varepsilon}{\mu} .
\end{split}
\end{equation}
\begin{equation*}
    \partial_{t'} \phi' + \frac{1}{2}\left(\frac{H_0^2}{L_1^2}\frac{\varepsilon}{\mu}(\partial_{x'}\phi')^2+\frac{\varepsilon}{\mu}(\partial_{z'}\phi')^2\right) + \frac{z'}{\varepsilon} = \frac{\mu}{\varepsilon}(P'-P'_{\text{atm}}).
\end{equation*}
Et comme $\frac{H_0^2}{L_1^2} = \mu$, on obtient finalement
\begin{equation}
    \partial_{t'} \phi' + \frac{\varepsilon}{2}\left((\partial_{x'}\phi')^2+\frac{1}{\mu}(\partial_{z'}\phi')^2\right) + \frac{z'}{\varepsilon} = \frac{\mu}{\varepsilon}(P'-P'_{\text{atm}}).\label{eq_B1_adim}
\end{equation}

\paragraph{Adimensionnement de (\ref{eq_B2}):} Après changement de variable, (\ref{eq_B2}) devient

\begin{equation*}
    \frac{(g t_0 a_{\text{surf}})}{L_1^2}\partial_{x' x'}\phi'+ \frac{(g t_0 a_{\text{surf}})}{H_0^2}\partial_{z' z'}\phi'  = 0
\end{equation*}

Après factorisation par $ \frac{g t_0 a_{\text{surf}} }{H_0^2}$, on voit apparaître $\mu$:
\begin{equation}
    \partial_{z'z'}\phi' + \mu \partial_{x_1' x_1'}\phi'  = 0 \label{eq_B2_adim}
\end{equation}



\paragraph{Adimensionnement de (\ref{eq_B3}):} Après changement de variable, dans $\{z' = \beta b'(x') - 1\}$, l'équation (\ref{eq_B3}) qui équivaut à  (\ref{eq_B3_bis}), devient

\begin{equation*}
    \left(\frac{a_{\text{bot}}}{L_1} \partial_{x'} b'\right)\left(\frac{gt_0a_{\text{surf}}}{L_1}\partial_{x'}\phi'\right)  - \frac{gt_0a_{\text{surf}}}{H_0}\partial_{z'}\phi' = 0 
\end{equation*}

Après factorisation par $\frac{g t_0 a_{\text{surf}}}{H_0} $, il vient
\begin{equation*}
    \frac{a_{\text{bot}}H_0}{L_1^2}  \partial_{x'} b' \partial_{x'}\phi'  - \partial_{z'}\phi' = 0 .
\end{equation*}

Autrement dit, dans $\{z' = \beta b'(x') - 1\}$, 
\begin{equation}
    \beta \mu \partial_{x'} b'  \partial_{x'}\phi'   - \partial_{z'}\phi' = 0 \label{eq_B3_adim}
\end{equation}

\paragraph{Adimensionnement de (\ref{eq_B4}):} Après changement de variable, dans $\{z = \zeta(x,t)\}$, l'équation (\ref{eq_B4}), qui équivaut à  (\ref{eq_B4_bis}), devient
\begin{equation*}
    \frac{a_{\text{surf}}}{t_0}\partial_{t'} \zeta' + \left(\frac{a_{\text{surf}}}{L_1} \partial_{x'} \zeta '\right)\left(\frac{gt_0a_{\text{surf}}}{L_1}\partial_{x'}\phi'\right)  - \frac{gt_0a_{\text{surf}}}{H_0}\partial_{z'}\phi' = 0 .
\end{equation*}
Après factorisation par $ \frac{a_{\text{surf}} }{t_0}$, on trouve
\begin{equation*}
     \partial_{t'}\zeta' +  \frac{gt_0^2a_{\text{surf}}}{L_1^2} \partial_{x'} \zeta ' \partial_{x'}\phi'  - \frac{gt_0^2}{H_0}\partial_{z'}\phi' = 0
\end{equation*}
Ce qui donne donc 
\begin{equation}
    \partial_{t'}\zeta' +  \varepsilon \partial_{x'} \zeta ' \partial_{x'}\phi'  - \frac{1}{\mu}\partial_{z'}\phi' = 0\label{eq_B4_adim}
\end{equation}


Par la suite, lorsqu'on travaillera sur les équations adimensionnées, on se permettra d'omettre les apostrophes sur ces variables.
\newpage
\section{Quelques régimes asymptotiques des équations d'Euler}
\subsection{Cas du fond plat 1D} 

Dans cette partie on s'intéressera au cas où $\textbf{X} = \mathbb{R}$ et $b = 0$. D'après, (\ref{eq_B1_adim}), (\ref{eq_B2_adim}), (\ref{eq_B3_adim}) et (\ref{eq_B4_adim}), on se retrouve alors à étudier le système suivant.

\begin{align}
~&\partial_{t} \phi + 
     \frac{\varepsilon}{2}(\partial_{x}\phi)^2+ \frac{\varepsilon}{2\mu}(\partial_{z}\phi)^2
     + \frac{z}{\varepsilon} 
     = \frac{\mu}{\varepsilon}(P-P_{\text{atm}}) &\text{ si } -1 < z < \varepsilon \zeta(t,x) \label{eq_k1}\\
~&\partial_{zz}\phi + \mu \partial_{x x}\phi  = 0  &\text{ si } -1 < z < \varepsilon \zeta(t,x) \label{eq_k2}\\
~&\partial_{z}\phi = 0&\text{ si } z  = - 1\label{eq_k3}\\
~&\partial_{t}\zeta +  \varepsilon \partial_{x} \zeta  \partial_{x}\phi  - \frac{1}{\mu}\partial_{z}\phi = 0 &\text{ si } z = \varepsilon \zeta(t,x)\label{eq_k4}
\end{align}

Plus précisément, on s'intéressera au cas d'une vague unidirectionnelle en supposant $\mu << 1$ et $\varepsilon<<1$.
\subsubsection{Développement limité de $\phi$ et premières conséquences}
Dans cette approche, proposée par \cite{MT}, on parvient à une simplification des équations d'Euler en remarquant pour commencer la proposition suivante
\begin{prop}
 
    Si $\phi$ est de classe $2n+2$ et vérifie (\ref{eq_k2}) et (\ref{eq_k3}), alors il existe une fonction $f:\mathbb{R}^+\times\textbf{X}\rightarrow \mathbb{R}$, de classe $2n+2$, telle que
    \begin{equation}
        \phi (t,x,z) = \sum_{j = 0}^{n} \mu^j\frac{(-1)^j}{(2j)!} (z+1)^{2j}\frac{\partial^{2j}f}{\partial x^{2j}}(x,t)
        + (-\mu)^{n+1}\int_{-1}^z\frac{(z-s)^{2n+2}}{(2n+2)!}\frac{\partial^{2n+2} \phi}{\partial x^{2n+2}}(t,x,s)ds
    \end{equation}
\end{prop}
\begin{proof}
Le développement de Taylor de $\phi$ en $z = -1$, à l'ordre $2n+2$, (avec reste intégrale) est
\begin{equation*}
\begin{split}
    \phi(t,x,z) = &\phi(t,x,-1) + (z+1)\frac{\partial \phi}{\partial z}(t,x,-1)\\
    &+ \frac{(z+1)^2}{2}\frac{\partial^2 \phi}{\partial z^2}(t,x,-1) + \frac{(z+1)^3}{3!}\frac{\partial^3 \phi}{\partial z^3}(t,x,-1)\\
    & \cdots\\
    &+ \frac{(z+1)^{2n}}{(2n)!}\frac{\partial^{2n} \phi}{\partial z^{2n}}(t,x,-1) + \frac{(z+1)^{2n+1}}{(2n+1)!}\frac{\partial^{2n+1} \phi}{\partial z^{2n+1}}(t,x,-1)\\
    &+\int_{-1}^z\frac{(z-s)^{2n+2}}{(2n+2)!}\frac{\partial^{2n+2} \phi}{\partial z^{2n+2}}(t,x,s)ds.
\end{split}
\end{equation*}

Par récurrence, l'équation (\ref{eq_k2}) nous donne, pour $z>-1$
\begin{equation*}
    \frac{\partial^{2j} \phi}{\partial z^{2j}}(t,x,z) = (-\mu)^{n}\frac{\partial^{2j} \phi}{\partial x^{2j}}(t,x,z)
\end{equation*}

\begin{equation*}
    \frac{\partial^{2j+1} \phi}{\partial z^{2j+1}}(t,x,z) = (-\mu)^{n}\frac{\partial^{2j}}{\partial x^{2j}}\frac{\partial \phi}{\partial z}(t,x,z)
\end{equation*}

En faisant tendre $z$ vers -1, il vient que les dérivées partielles $\frac{\partial^{k} \phi}{\partial z^{k}}(t,x,z)$ sont bien définies en $z = -1$ et 
\begin{equation*}
    \frac{\partial^{2j} \phi}{\partial z^{2j}}(t,x,-1) = (-\mu)^{n}\frac{\partial^{2j} \phi}{\partial x^{2j}}(t,x,-1)
\end{equation*}
\begin{equation*}
    \frac{\partial^{2j+1} \phi}{\partial z^{2j+1}}(t,x,-1) = (-\mu)^{n}\frac{\partial^{2j}}{\partial x^{2j}}\frac{\partial \phi}{\partial z}(t,x,-1)
\end{equation*}

De plus, l'équation (\ref{eq_k3}), nous donne $\frac{\partial \phi}{\partial z}(t,x,-1) = 0$ pour tout $x\in \textbf{X}$ et donc 
\begin{equation*}
    \frac{\partial^{2j}}{\partial x^{2j}}\frac{\partial \phi}{\partial z}(t,x,-1) = 0
\end{equation*}

Finalement, on obtient

    \begin{equation*}
        \phi (t,x,z) = \sum_{j = 0}^{n} \mu^j\frac{(-1)^j}{(2j)!} (z+1)^{2j}\frac{\partial^{2j}\phi}{\partial x^{2j}}(t,x,-1)
        + (-\mu)^{n+1}\int_{-1}^z\frac{(z-s)^{2n+2}}{(2n+2)!}\frac{\partial^{2n+2} \phi}{\partial x^{2n+2}}(t,x,s)ds
    \end{equation*}


Ce qui, si on pose $f(t,x) = \phi(t,x,-1)$, nous donne le résultat.
\end{proof}

Ainsi, on se retrouve gràce à l'hypothèse ($\textbf{H}_9$) avec les approximation suivantes.

\begin{equation}
    \begin{split}
            &\phi = f - \mu\frac{(z+1)^2}{2}\partial_{xx}f + \mu^2\frac{(z+1)^4}{24}\partial_{xxxx}f + \mathcal{O}(\mu^3)\\
            &\partial_z\phi = -\mu(z+1)\partial_{xx}f + \mu^2\frac{(z+1)^3}{6}\partial_{xxxx}f + \mathcal{O}(\mu^3)\\
            &\partial_x\phi = \partial_xf - \mu\frac{(z+1)^2}{2}\partial_{xxx}f + \mathcal{O}(\mu^2)
    \end{split}
\end{equation}
\paragraph{En injectant ces équations dans (\ref{eq_k4}):}
Lorsque $z = \varepsilon \zeta(t,x)$, on se retrouve avec l'expression suivante
\begin{equation*}
    \partial_{t}\zeta +  \varepsilon \partial_{x} \zeta  \left[\partial_xf - \mu\frac{(z+1)^2}{2}\partial_{xxx}f + \mathcal{O}(\mu^2) \right]  +  (z+1) \partial_{xx}f - \mu\frac{(z+1)^3}{6}\partial_{xxxx}f + \mathcal{O}(\mu^2) = 0
\end{equation*}
que l'on peut réécrire ainsi.
\begin{equation*}
    \partial_{t}\zeta +  \partial_{x}(\varepsilon \zeta+1) \partial_xf+  (\varepsilon\zeta+1) \partial_{xx}f    - \mu\frac{(\varepsilon\zeta+1)^3}{6}\partial_{xxxx}f - \mu \varepsilon\partial_{x}\zeta\frac{(\varepsilon\zeta+1)^2}{2}\partial_{xxx}f= \mathcal{O}(\mu^2) + \mathcal{O}(\varepsilon\mu^2) 
\end{equation*}

Où encore

\begin{equation}
    \partial_{t}\zeta +  \partial_{x}((\varepsilon \zeta+1) \partial_xf)    - \mu\frac{(\varepsilon\zeta+1)^3}{6}\partial_{xxxx}f - \mu \varepsilon\partial_{x}\zeta\frac{(\varepsilon\zeta+1)^2}{2}\partial_{xxx}f= \mathcal{O}(\mu^2) + \mathcal{O}(\varepsilon\mu^2) \label{eq_approx1} 
\end{equation}
\paragraph{En injectant ces équations dans (\ref{eq_k1}):} Lorsque $z = \varepsilon \zeta(t,x)$, on se retrouve avec $P -P_{atm} = 0$ et il vient alors l'expression suivante.

\begin{equation*}
\begin{split}
    \partial_{t}\left[f - \mu\frac{(z+1)^2}{2}\partial_{xx}f + \mathcal{O}(\mu^2)\right] 
    &+\frac{\varepsilon}{2}\left(\partial_xf - \mu\frac{(z+1)^2}{2}\partial_{xxx}f + \mathcal{O}(\mu^2)\right)^2 
     \\&+\frac{\varepsilon}{2\mu}\left(\mu(z+1)\partial_{xx}f + \mu^2\frac{(z+1)^3}{6}\partial_{xxxx}f + \mathcal{O}(\mu^3)\right)^2
     \\&+ \frac{z}{\varepsilon} 
     = 0
\end{split}
\end{equation*}
que l'on peut réécrire
\begin{equation}
\begin{split}
   \left[ \partial_{t}f - \mu\frac{(\varepsilon\zeta+1)^2}{2}\partial_{xxt}f + \mathcal{O}(\mu^2)\right] 
    &+\frac{\varepsilon}{2}\left(  (\partial_xf)^2 -  \mu(\varepsilon\zeta+1)^2 \partial_xf\partial_{xxx}f \right) + \mathcal{O}(\varepsilon\mu^2)   
     \\&+\frac{\varepsilon\mu}{2}(\varepsilon\zeta+1)^2(\partial_{xx}f)^2 +  \mathcal{O}(\varepsilon\mu^2) 
     \\&+  \zeta 
     = 0
\end{split}\label{eq_approx2} 
\end{equation}



\subsubsection{Vers les équations de Korteweg-de Vries.}
A partir de (\ref{eq_approx2}), \cite{MT} détaille l'obtention d'une nouvelle équation basée sur les 3 étapes suivante:
\begin{center}

    --------------------------------------------\\
   \textit{Premièrement, on néglige tous les termes d'ordre 1 ou plus en $\mu$ et en $\varepsilon$}\\
   \textit{"on remplace $\mathcal{O}(\mu,\varepsilon)$ par 0"}\\
   --------------------------------------------
\end{center}

Si $\mu = \varepsilon = 0$ dans (\ref{eq_approx1}) et (\ref{eq_approx2}) on obtient le système suivant

\begin{align}
    &\partial_t\zeta + \partial_{xx}f = 0 \label{eq_prekdv_1}  \tag{$\texttt{h}_1^{\mathcal{O}^0}$}\\
    &\partial_tf + \zeta = 0 \label{eq_prekdv_2} \tag{$\texttt{h}_2^{\mathcal{O}^0}$}
\end{align}

En combinant (\ref{eq_approx2}) et la dérivée de (\ref{eq_prekdv_1}) par rapport à $t$, on trouve

\begin{align}
    &\partial_{tt} \zeta - \partial_{xx}\zeta = 0 \label{eq_wave}\\
    &\partial_{xx} \zeta + \partial_{xx}f = 0 \label{eq_duality}
\end{align}

L'équation (\ref{eq_wave}) s'appelle équation des ondes. Une solution unidirectionnelle de cette équation est de la forme $\zeta(x,t) = g(x-t)$. Ce qui nous donne
\begin{equation*}
    \partial_x\zeta = - \partial_t \zeta 
\end{equation*}
De cette équation et à partir de (\ref{eq_prekdv_1}) et (\ref{eq_prekdv_2}),  on obtient que 
\begin{align*}
   &\partial_x(\partial_xf-\zeta) = \partial_{xx}f+\partial_t\zeta = 0\\
   &\partial_t(\partial_xf-\zeta) = \partial_{xt}f+\partial_x\zeta = \partial_{x}(\partial_{t}f+\zeta) = 0
\end{align*}
Autrement dit, $\partial_xf-\zeta = 0$

\begin{center}
    --------------------------------------------\\
    \textit{Deuxièmement, on corrige le résultat obtenu en ajoutant des termes d'un ordre supérieur.}\\
   \textit{"on remplace 0 par $\mathcal{O}(\mu,\varepsilon)$"}\\
   --------------------------------------------
\end{center}
Supposons désormais vraies les assertions suivantes
\begin{align}
    &\partial_xf - \zeta = \mu A + \varepsilon B + \mathcal{O}(\mu^2+\varepsilon^2) \label{hyp_prekdv1} \tag{$\texttt{h}_1^{\mathcal{O}^1}$} \\
    & \partial_x \zeta + \partial_t\zeta = \mathcal{O}(\mu, \varepsilon) \label{hyp_prekdv2} \tag{$\texttt{h}_2^{\mathcal{O}^1}$}
\end{align}

Si on dérive (\ref{eq_approx2}) par rapport à $x$ et si on néglige les termes de l'ordre de $\mu^2$, de $\mu\varepsilon$ et de $\varepsilon^2$ dans (\ref{eq_approx1}) et (\ref{eq_approx2}), on obtient le système suivant
\begin{align*}
    &\partial_t\zeta+\partial_x((\varepsilon\zeta + 1)\partial_xf) - \mu\frac{1}{6}\partial_{xxxx}f = \mathcal{O}(\mu^2, \mu\varepsilon, \varepsilon^2)\\
    &\partial_{xt}f - \mu \frac{1}{2}\partial_{xxxt}f + \varepsilon\partial_xf\partial_{xx}f + \zeta_x = \mathcal{O}(\mu^2, \mu\varepsilon, \varepsilon^2)
\end{align*}

En remplaçant les dérivées de $f$ de ce système respectivement à l'hypothèse (\ref{hyp_prekdv1}) et en continuant de négliger les termes d'ordre strictement supérieur à 1 en $\mu$ et $\varepsilon$, il vient le système suivant

\begin{align*}
    &\partial_t\zeta+\partial_x((\varepsilon\zeta + 1)\zeta) + \partial_x(\mu A + \varepsilon B) - \mu\frac{1}{6}\partial_{xxx}\zeta = \mathcal{O}(\mu^2, \mu\varepsilon, \varepsilon^2)\\
    &\partial_t \zeta + \mu\partial_tA+\varepsilon\partial_tB - \mu \frac{1}{2}\partial_{xxt}\zeta + \varepsilon\zeta\partial_{x}\zeta + \partial_x \zeta = \mathcal{O}(\mu^2, \mu\varepsilon, \varepsilon^2)
\end{align*}

Toujours dans l'optique de négliger les termes d'ordre élevés, lorsque $\mu$ ou $\varepsilon$ est en facteur d'une dérivée temporelle de $\zeta$ on peut la remplacer par la dérivée spatiale correspondante grâce à l'hypothèse (\ref{hyp_prekdv2}). Ce qui nous amène au système suivant
\begin{align*}
    &\partial_t\zeta +\partial_x((\varepsilon\zeta + 1)\zeta) + \partial_x(\mu A + \varepsilon B) - \mu\frac{1}{6}\partial_{xxx}\zeta = \mathcal{O}(\mu^2, \mu\varepsilon, \varepsilon^2)\\
    &\partial_t \zeta + \mu\partial_tA+\varepsilon\partial_tB + \mu \frac{1}{2}\partial_{xxx}\zeta + \varepsilon\zeta\partial_{x}\zeta + \partial_x \zeta = \mathcal{O}(\mu^2, \mu\varepsilon, \varepsilon^2)
\end{align*}

On peut réécrire ce système ainsi
\begin{align}
    &\partial_t\zeta + \partial_x\zeta + \mu  \big( \partial_x A - \frac{1}{6}\partial_{xxx}\zeta \big) + \varepsilon  \big( 2\zeta\partial_x\zeta + \partial_x B \big) = \mathcal{O}(\mu^2, \mu\varepsilon, \varepsilon^2) \label{eq_withAandB} \\
    &\partial_t\zeta + \partial_x\zeta + \mu  \big( \partial_t A + \frac{1}{2}\partial_{xxx}\zeta \big) + \varepsilon  \big(  \zeta\partial_x\zeta + \partial_t B \big) = \mathcal{O}(\mu^2, \mu\varepsilon, \varepsilon^2)\notag
\end{align}

Par unicité [\texttt{Argument discutable}] du développement asymptotique, on a alors
\begin{align*}
    & \partial_x A - \frac{1}{6}\partial_{xxx}\zeta = \partial_t A + \frac{1}{2}\partial_{xxx}\zeta + \mathcal{O}(\mu, \varepsilon)\\
    &2\zeta\partial_x\zeta + \partial_x B =\zeta\partial_x\zeta + \partial_t B + \mathcal{O}(\mu, \varepsilon)
\end{align*}

On cherche deux fonction $A$ et $B$ qui conviennent, c'est à dire deux fonctions vérifiant 
\begin{align}
    & \partial_x A -\partial_t A =  \frac{2}{3}\partial_{xxx}\zeta  + \mathcal{O}(\mu, \varepsilon)\label{eq_A_tosolve}\\
    &\partial_t B - \partial_x B =\zeta\partial_x\zeta  + \mathcal{O}(\mu, \varepsilon)\label{eq_B_tosolve}
\end{align}

Supposant connu $\zeta$, résolvons (\ref{eq_A_tosolve}) et (\ref{eq_B_tosolve}) à l'aide du changement de variable $s = x+t$, et $r = x-t$:

Ce qui donne pour toute fonction $d\in \{\zeta,A,B\}$
\begin{align}
    \tilde{d}(s,r) &= d\left(\frac{s+r}{2}, \frac{s-r}{2}\right)\\
    \partial_s\tilde{d}(s,r) &= \frac{1}{2}\left(\partial_xd + \partial_td\right)\left(\frac{s+r}{2}, \frac{s-r}{2}\right)\\
    \partial_r\tilde{d}(s,r) &= \frac{1}{2} \left(\partial_xd - \partial_td\right)\left(\frac{s+r}{2}, \frac{s-r}{2}\right)\label{eq_dxdr1}
\end{align}

D'après (\ref{hyp_prekdv2}), on trouve alors $\partial_s\tilde{\zeta} = \mathcal{O}(\mu, \varepsilon)$, et comme $\partial_x\zeta = (\partial_s\tilde{\zeta} + \partial_r\tilde{\zeta})(x+t, x-t)$ il vient alors que
\begin{equation}
    \partial_x\zeta = \partial_r\tilde{\zeta}(x+t,x-t) + \mathcal{O}(\mu,\varepsilon) \label{eq_dxdr2}
\end{equation}

En injectant (\ref{eq_dxdr1}), (\ref{eq_dxdr2}) dans (\ref{eq_A_tosolve}) et dans (\ref{eq_B_tosolve}) il vient
\begin{align*}
    2\partial_r\tilde{A} &= \frac{2}{3}\partial_{rrr}\tilde\zeta  +\mathcal{O}(\mu,\varepsilon)\\
    -2\partial_r\tilde{B} &= \tilde\zeta\partial_r\tilde\zeta +\mathcal{O}(\mu,\varepsilon)
\end{align*}

On peut donc prendre 

\begin{align*}
    \tilde{A} &= \frac{1}{3}\partial_{rr}\tilde\zeta \\
    \tilde{B} &= \frac{-\tilde\zeta^2}{4}
\end{align*}

et donc 
\begin{align*}
    A &= \frac{1}{3}\partial_{xx}\zeta \\
    B &= \frac{-\zeta^2}{4}
\end{align*}
Finalement l'équation (\ref{eq_withAandB}) devient

\begin{equation*}
    \partial_t\zeta + \partial_x\zeta + \mu  \frac{1}{6}\partial_{xxx}\zeta + \varepsilon  \frac{2}{3}\zeta\partial_x\zeta  = \mathcal{O}(\mu^2, \mu\varepsilon, \varepsilon^2)
\end{equation*}

\begin{center}
    --------------------------------------------\\
   \textit{Finalement, on néglige tous les termes d'ordre 2 ou plus en $\mu$ et en $\varepsilon$}\\
   \textit{"on remplace $\mathcal{O}(\mu^2, \mu\varepsilon,\varepsilon^2)$ par 0"}\\
   --------------------------------------------
\end{center}

En considérant l'équation 

\begin{equation}
    \partial_t\zeta + \partial_x\zeta + \mu  \frac{1}{6}\partial_{xxx}\zeta + \varepsilon  \frac{2}{3}\zeta\partial_x\zeta  = 0 \label{eq_KdV_beforechvar}
\end{equation}

Par un changement de variable on peut se ramener à l'étude d'une équation ne dépendant plus des paramètre $\mu$ et $\varepsilon$ . En effet, en posant $u(x,t) = \zeta(kx-lt,lt)$ , il vient

\begin{equation*}
    \frac{1}{l}\partial_tu + \mu  \frac{1}{6k^3}\partial_{xxx}u + \varepsilon  \frac{2}{3k}u\partial_xu  = 0
\end{equation*}

Et en posant $k = \dfrac{\sqrt{\mu}}{2\varepsilon}$ et $ l =\dfrac{3\sqrt{\mu}}{4\varepsilon^2}$, on obtient finalement les équations de KdV


\begin{equation}
    \partial_tu + \partial_{xxx}u +u\partial_xu  = 0 \label{KdV}\tag{\textbf{KdV}}
\end{equation}
\paragraph{Remarque:} Dans le cas d'une vague de très faible amplitude ($1>>\mu>>\varepsilon$), le troisième terme de (\ref{eq_KdV_beforechvar}) disparaît.  Avec un changement de variable analogue à l'obtention de (\ref{KdV}), on obtient alors les équations de Airy.

\begin{equation}
    \partial_tu + \partial_{xxx}u = 0 \label{Airy}\tag{\textbf{Airy}}
\end{equation}

\subsection{Cas d'un fond fixe non plat.} 
\subsubsection{Mise en place des hypothèses}
Dans cette partie on s'intéressera au cas où $\textbf{X} = \mathbb{R}$ mais où $b$ n'est plus nécessairement nul. De plus, nous ajoutons l'hypothèse suivante appelée hypothèse hydrostatique:
\begin{equation}
    P = P_\text{atm} + g\rho(\zeta - z)
\end{equation}
\subsubsection{Petit retour aux équations dimensionnées}
On rappelle que $\textbf{U} = \nabla \phi = \left(\begin{array}{cc}
     \partial_x\phi\\
     \partial_z\phi
\end{array}\right)$ où $\textbf{U}$ est le champ de vitesse du fluide. On pose $$\overline{\textbf{U}}(t,x) = \frac{1}{h(x)}\int_{b(x)-H_0}^{\zeta(t,x)} \textbf{U}(t,x,z)dz$$
où $h(t,x) = \zeta(t,x) - b(x) + H_0$. Alors, $\overline{U}$ désigne la vitesse moyenne du fluide le long de l'axe $z$. En particulier, nous nous intéresserons à la première composante $\overline{\textbf{U}}_1$ de cette moyenne.

Après adimensionnement, on se retrouve avec 
$$\overline{\textbf{U}}_1'(t',x') :=\frac{1}{h'(t',x')}\int_{\beta b'(x')-1}^{\varepsilon\zeta'(t',x')}\partial_{x'}\phi'dz'= \frac{1}{gt_0\varepsilon\sqrt{\mu}}\overline{\textbf{U}}_1(t_0t',L_1x')  $$
où $ h' =  \varepsilon\zeta' - \beta b' + 1$

Les propositions qui vont suivre permettent de faire le lien entre cette vitesse moyenne et les différents paramètres de nos équations. Dans la première tirée d'une démonstration présente dans \cite{Lannes}, l'utilisation de la formule de Green permet de facilement généraliser ce résultat au cas de la dimensions 2.
\begin{prop} \label{prop:meanspeed_to_eq4}
    $$\partial_{z'}\phi - \mu\varepsilon\partial_{x'}\zeta'\partial_{x'}\phi' = -\mu \partial_{x'}\left(h'\overline{\textbf{U}}'_1 \right)$$
\end{prop}
\,\\
\begin{proof}
    D'après la formule de Green, 

    \begin{equation}
        \int_{\partial\Omega_t}{\partial_n\phi \times v} = \int_{\Omega_t}{\nabla\phi \cdot \nabla v}~ + \int_{\Omega_t}{v\Delta \phi}
    \end{equation}

    Autrement dit
    \begin{equation}
        \int_{\{z = \zeta(x)\}}\partial_n\phi \times vd\sigma + \int_{\{z = b(x) - H_0\}}{\partial_n\phi \times v}d\sigma = \int_{\Omega_t}{\nabla\phi \cdot \nabla v}~+ \int_{\Omega_t}{v\Delta \phi}
    \end{equation}

    Que l'on peut aussi écrire
    \begin{equation}
    \begin{split}
        \int_{-\infty}^\infty\partial_n\phi_{|_{\{z = \zeta(x)\}}} \times v \sqrt{1+|\partial_x\zeta|^2}dx &- \int_{-\infty}^\infty\partial_n\phi_{|_{\{z = b(x)-H_0\}}} \times v \sqrt{1+|\partial_xb|^2}dx \\&= \int_{\Omega_t}{\nabla\phi \cdot \nabla v}~+ \int_{\Omega_t}{v\Delta \phi}
    \end{split}
    \end{equation}
    
    D'après (\ref{eq_B2}) (resp. (\ref{eq_B3})), le quatrième terme (resp. le second terme) est nul. On se retrouve alors avec
    \begin{equation}
        \int_{-\infty}^\infty\partial_n\phi_{|_{\{z = \zeta(x)\}}}  \times v \sqrt{1+|\partial_x\zeta|^2}dx = \int_{\Omega_t}{\nabla\phi \cdot \nabla v}~
    \end{equation}
    
    On rappelle que
    \begin{align}
        \sqrt{1+ |\partial_x\zeta|^2}\partial_\textbf{n}\phi_{|_{\{z = \zeta(x)\}}} (t_0t,L_1x) &= \left(\partial_z\phi -\partial_x\zeta\partial_x\phi\right) (t_0t,L_1x)\\
            &=\frac{gt_0a_{\text{surf}}}{H_0}\partial_{z'}\phi' - \left(\frac{a_{\text{surf}}}{L_1} 
            \partial_{x'} \zeta '\right)\left(\frac{gt_0a_{\text{surf}}}{L_1}\partial_{x'}\phi'\right)\\
            &=gt_0\left( \varepsilon\partial_{z'}\phi' - \frac{a_{\text{surf}}^2}{L_1^2}\left( \partial_{x'} \zeta '\partial_{x'}\phi'\right)\right)\\
            &=gt_0\varepsilon\left(\partial_{z'}\phi' -\mu\varepsilon \partial_{x'} \zeta '\partial_{x'}\phi'\right)
    \end{align}

    Supposons maintenant que $v$ ne dépend pas de la variable $z$. Ainsi
    \begin{equation}
        \begin{split}
            \int_{\Omega_t}{\nabla\phi \cdot \nabla v} &= \int_{\Omega_t}{\partial_x\phi \partial_xv}\\
            &= \int_{-\infty}^{+\infty} \partial_xv \left(\int_{b(x) - H_0}^{\zeta(x)}\partial_x\phi dz \right) ~dx\\
            &= \int_{-\infty}^{+\infty}\partial_xv \times h\overline{\textbf{U}}_1 dx\\
            &= \int_{-\infty}^{+\infty} \partial_xv \times \left(H_0h'\times gt_0\varepsilon\sqrt{\mu}\overline{\textbf{U}}'_1 \right)\left(\frac{t}{t_0},\frac{x}{L_1}\right)dx\\
            \textit{(i.p.p.)}~~~&= -\frac{1}{L_1}\int_{-\infty}^{+\infty} v \times \partial_{x'}\left(H_0h'\times gt_0\varepsilon\sqrt{\mu}\overline{\textbf{U}}'_1 \right)\left(\frac{t}{t_0},\frac{x}{L_1}\right)dx\\
            &= - \mu g t_0 \varepsilon\int_{-\infty}^{+\infty} v \times \partial_{x'}\left(h'\times \overline{\textbf{U}}'_1 \right)\left(\frac{t}{t_0},\frac{x}{L_1}\right)dx
        \end{split}
    \end{equation}
    On se retrouve alors avec l'égalité
    \begin{equation*}
         gt_0\varepsilon\int_{-\infty}^\infty v \times \left(\partial_{z'}\phi' -\mu\varepsilon \partial_{x'} \zeta '\partial_{x'}\phi'\right)= -g t_0 \varepsilon\int_{-\infty}^{+\infty} v \times \mu \partial_{x'}\left(h'\times \overline{\textbf{U}}'_1 \right)\left(\frac{t}{t_0},\frac{x}{L_1}\right)dx
    \end{equation*}
    et ce pour tout $v\in\mathcal{C}^\infty(\textbf{X})$ ceci nous donne le résultat.
\end{proof}
\begin{prop} \label{hydrostatic_hyp_adim}
    L'hypothèse hydrostatique adimensionnée est
    \begin{equation}
        P' = P_\text{atm}' + \frac{1}{\mu}(\varepsilon\zeta' - z')
    \end{equation}
\end{prop}

\begin{proof}
    \begin{equation*}
        \begin{split}
            0 &= P - P_\text{atm} - g\rho_0(\zeta - z)\\
            &= \frac{H_0^2\rho_0}{t_0^2}\left(P' - P_\text{atm}'\right) - g\rho_0\left(a_\text{surf}\zeta' - H_0z'\right)\\
            &= \left(P' - P_\text{atm}'\right) - \frac{t_0^2g}{H_0^2}\left(a_\text{surf}\zeta' - H_0z'\right)\\
            &= \left(P' - P_\text{atm}'\right) - \frac{t_0^2g}{H_0}\left(\varepsilon\zeta' - z'\right)\\
            &= \left(P' - P_\text{atm}'\right) -\frac{1}{\mu}\left(\varepsilon\zeta' - z'\right)\\
        \end{split}
    \end{equation*}
\end{proof}
\subsubsection{Vers les équations de Saint Venant}

D'après, (\ref{eq_B1_adim}), (\ref{eq_B2_adim}), (\ref{eq_B3_adim}) et (\ref{eq_B4_adim}), on se retrouve alors à étudier le système suivant.
\begin{align}
~&\partial_{t} \phi + 
     \frac{\varepsilon}{2}(\partial_{x}\phi)^2+ \frac{\varepsilon}{2\mu}(\partial_{z}\phi)^2
     + \frac{z}{\varepsilon} 
     = \frac{\mu}{\varepsilon}(P-P_\text{atm}) &\text{ si } \beta b(x) -1 < z < \varepsilon \zeta(t,x) \label{eq_1k1}\tag{$\textbf{B}^{\text{adim}}_1$}\\
~&\partial_{zz}\phi + \mu \partial_{x x}\phi  = 0  &\text{ si } \beta b(x) -1 < z < \varepsilon \zeta(t,x) \label{eq_1k2}\tag{$\textbf{B}^{\text{adim}}_2$}\\
~&\beta \mu \partial_{x} b  \partial_{x}\phi   - \partial_{z}\phi = 0& \text{dans }\{z = \beta b(x) - 1\}\label{eq_1k3} \tag{$\textbf{B}^{\text{adim}}_3$}\\
~&\partial_{t}\zeta +  \varepsilon \partial_{x} \zeta  \partial_{x}\phi  - \frac{1}{\mu}\partial_{z}\phi = 0 &\text{ si } z = \varepsilon \zeta(t,x)\label{eq_1k4} \tag{$\textbf{B}^{\text{adim}}_4$}
\end{align}

\begin{prop}
    Si on suppose l'hypothèse hydrostatique vérifiée et que $\mu<<1$, alors 
    \begin{align}
        &\partial_th + \varepsilon\partial_{x}(h\overline{U}_1) = 0\\
        &\partial_t\overline{U}_1+\frac{\varepsilon}{2}\partial_x{U_1^2} + \partial_xh  = - \beta\partial_x b +  \mathcal{O}(\mu)
    \end{align}
\end{prop}
\begin{proof}
    D'après l'hypothèse hydrostatique (i.e. la proposition \ref{hydrostatic_hyp_adim}),  dans $\{z \in ]\beta b(x) -1 , \varepsilon \zeta(t,x) [\}$,  l'équation (\ref{eq_1k1}) peut aussi s'écrire
\begin{equation}
  \partial_{t} \phi + 
     \frac{\varepsilon}{2}(\partial_{x}\phi)^2+ \frac{\varepsilon}{2\mu}(\partial_{z}\phi)^2
     = -\zeta   \label{eq_1K1_h}
\end{equation}

D'après la proposition \ref{prop:meanspeed_to_eq4}, et comme $\partial_t\zeta = \frac{1}{\varepsilon}\partial_th$, l'équation (\ref{eq_1k4}) peut aussi s'écrire

\begin{equation}
  \partial_th + \varepsilon\partial_{x}(h\overline{U}_1) = 0
\end{equation}




De plus, les équations (\ref{eq_1k2}), (\ref{eq_1k3}) et (\ref{eq_1k4}) nous donnent
\begin{align}
    &\partial_z \phi = \mathcal{O}(\mu) \text{ dans } \partial\Omega_t \label{dzphismall}\\
    &\partial_{zz} \phi = \mathcal{O}(\mu) \text{ dans } \Omega_t
\end{align}

En dérivant $\partial_{zz}\phi$ par rapport à $x$ puis en intégrant par rapport à $z$, il vient



\begin{equation}
    \partial_{z}\textbf{U}_1 = \mathcal{O}(\mu)
\end{equation}

Un développement de Taylor à l'ordre 1 en $z_0 \in [\beta b (x) - 1 , \varepsilon \zeta(x)]$ donne alors
\begin{equation}
    \textbf{U}_1(t,x,z) = \textbf{U}_1(t,x,z_0) + \mathcal{O}(\mu)
\end{equation}

En intégrant par rapport à $z_0$, il vient alors
\begin{equation}
    \textbf{U}_1(t,x,z) = \overline{\textbf{U}_1}(t,x) + \mathcal{O}(\mu)
\end{equation}

De plus en injectant (\ref{dzphismall}) dans (\ref{eq_1k1}) on se retrouve avec
\begin{equation}
    \partial_{t} \phi + 
     \frac{\varepsilon}{2}(\partial_{x}\phi)^2+ \mathcal{O}(\varepsilon\mu) 
     =  -\zeta
\end{equation}

En dérivant ce résultat par rapport à $x$, on a alors
\begin{equation}
    \partial_{tx}\phi  + 
     \varepsilon\partial_{xx}\phi\partial_{x}\phi+ \mathcal{O}(\varepsilon\mu)
     = -\partial_x\zeta \label{tobeintegrate}
\end{equation}



Ainsi, en intégrant (\ref{tobeintegrate}) par rapport à $z$ et en divisant par $h$, on trouve
\begin{equation}
    \partial_t\overline{U}_1+\frac{\varepsilon}{2}\partial_x{U_1^2} + \partial_x\zeta = \mathcal{O}(\mu)
\end{equation}
\end{proof}
\newpage
\section{Étude numérique de ces quelques équations}
\subsection{ L'équation de Airy}
L'équation (\ref{Airy}) est une équation linéaire simple à résoudre. Il est toutefois très intéressant d'étudier les différents schémas numériques qui permettent de modéliser cette équation d'évolution. Pour cause, connaître l'action de ces schémas sur cette équation nous permettra de connaître, avec précision, l'effet de nos futurs schémas sur le terme de dispersion $\partial_{xxx}u$ de l'équation de (\ref{KdV}).
\\
\subsubsection{Analyse qualitative de l'équation}
En connaissant la condition initiale $u(\cdot,0) = u_0$, on connaît, analytiquement, la solution $u$ en tout temps. En effet, si $\hat{u}$ désigne la transformée de Fourier de $u$, l'équation (\ref{Airy}) devient

\begin{equation*}
    \partial_t\hat{u}(\xi, t) - i\xi^3\hat{u}(\xi, t) = 0.
\end{equation*}

Ceci nous donne alors
\begin{equation*}
    \hat{u}(\xi, t) = e^{it\xi^3}\hat{u_0}(\xi).
\end{equation*}

Par transformée de Fourier inverse, on obtient alors une expression pour $u$ en tout temps. De plus, deux conséquences se déduisent de cette expression.
\paragraph{Première conséquence :} Il y a la conservation de la norme $L^2$ de $u$,
\begin{equation}
    \|u\|_{L^2}(t) = \|\hat{u}\|_{L^2}(t) =\|e^{it\xi^3}\hat{u_0}\|_{L^2} =  \|\hat{u_0}\|_{L^2} =  \|u_0\|_{L^2} .
\end{equation}

Nous nous intéresserons alors à l'utilisation de schémas qui conservent cette norme de sorte qu'il n'y ait ni dissipation d'énergie, ni création d'instabilité.

\paragraph{Deuxième conséquence :} Les harmoniques de $u$ son conservées et se déplacent à vitesse $v_\xi$ constante en temps dépendant de la fréquence $\xi$. La translation d'une harmonique se traduit par le déphasage du coefficients de Fourier en espace associé. Plus précisément, 

\begin{equation}
\begin{split}
    v_\xi &= \frac{1}{\xi}\partial_t (\text{Arg}(\hat{u}(\xi,\cdot)))
    \\ &= \frac{1}{\xi}\partial_t (t\xi^3) 
    \\ &= \xi^2
\end{split}
\end{equation}

Nous essaierons alors au mieux de nous concentrer sur l'utilisation de schémas apportant une vitesse des harmoniques la plus fidèle possible.

\subsubsection{ Discrétisation en temps: }
Commençons par discrétiser la variable temporelle uniformément. Posons $t^n = n\Delta_t$, et intéressons nous à l'approximation $\texttt{U}^n \approx u(\cdot,t^n)$ générée par le $\theta-$schéma si dessous.
\begin{equation}
\frac{\texttt{U}^{n+1}-\texttt{U}^n}{\Delta_t} + \partial_{xxx}(\theta\texttt{U}^{n+1} + (1-\theta)\texttt{U}^{n}) = 0 \label{num_theta_scheme_Airy}
\end{equation}

Les transformées de Fourier $\widehat{\texttt{U}^n}$ vérifient alors
\begin{equation*}
\frac{\widehat{\texttt{U}^{n+1}}-\widehat{\texttt{U}^n}}{\Delta_t} - i \xi^3(\theta\widehat{\texttt{U}^{n+1}} + (1-\theta)\widehat{\texttt{U}^{n}}) = 0 
\end{equation*}

Ce qui donne 
\begin{equation*}
\widehat{\texttt{U}^{n+1}} = \frac{1 + i (1-\theta)\Delta_t\xi^3}{1 - i\theta \Delta_t\xi^3 }\widehat{\texttt{U}^{n}}  
\end{equation*}

\paragraph{Étude de la conservation de la norme $L^2$ de notre schéma}\,
\\

Si $\theta < \frac{1}{2}$, on a l'inégalité $\left|\frac{1 + i (1-\theta)\Delta_t\xi^3}{1 - i\theta \Delta_t\xi^3 }\right|>1 $  et donc le $\theta-$schéma est instable au sens de la norme $L^2$.\\

Si $\theta > \frac{1}{2}$, on a l'inégalité $\left|\frac{1 + i (1-\theta)\Delta_t\xi^3}{1 - i\theta \Delta_t\xi^3 }\right|<1 $ et donc la norme $L^2$ décroît.\\

Si $\theta = \frac{1}{2}$, on a l'inégalité $\left|\frac{1 + i (1-\theta)\Delta_t\xi^3}{1 - i\theta \Delta_t\xi^3 }\right| = 1$ et donc la norme $L^2$ est conservée, c'est le schéma de Crank-Nicholson.\\

\paragraph{Étude de l'évolution des harmoniques de $\texttt{U}^n$:} 
On souhaite avoir une estimation de la vitesse de chaque harmonique de $\texttt{U}^n$, pour cela nous allons approcher $v_\xi$ par $ \texttt{v}_\xi^{\Delta_t}$ définie par
\begin{equation}
\begin{split}
     \texttt{v}_\xi^{\Delta_t} &= \frac{1}{\xi}\left( \frac{\text{Arg}(\widehat{\texttt{U}}^{n+1}) - \text{Arg}(\widehat{\texttt{U}}^{n})}{\Delta_t} \right)
    \\ & =\frac{1}{\Delta_t\xi}\left(\text{Arg}\left( \frac{\widehat{\texttt{U}}^{n+1}}{\widehat{\texttt{U}}^{n}}\right)\right) 
    \\ & = \frac{1}{\Delta_t\xi}\left(\text{Arg}\left(\frac{1 + i (1-\theta)\Delta_t\xi^3}{1 - i\theta \Delta_t\xi^3 }\right)\right)
    \\ & = \frac{1}{\Delta_t\xi}\left(\text{Arg}\left(1 + i (1-\theta)\Delta_t\xi^3\right)- \text{Arg}\left(1 - i\theta \Delta_t\xi^3 \right)\right)
    \\ & = \frac{1}{\Delta_t\xi}\left(\text{Arctan}((1-\theta)\Delta_t\xi^3)+\text{Arctan}(\theta\Delta_t\xi^3)\right)
\end{split}
\end{equation}

Le développement limité de Arctan en 0 nous donne alors

\begin{equation}
    \texttt{v}_\xi^{\Delta_t} = \xi^2 - \frac{1}{3}\left((1-\theta)^3+\theta^3\right)\Delta_t^2\xi^8 + \mathcal{O}((\Delta_t)^4) 
\end{equation}

Pour la suite, on supposera que l'ensemble $Supp(\hat{u}):=\{\xi > 0 ~|~ \|\hat{u}\|>0\}$ est borné (majoré par une constante $\xi_\text{max}$). On s'assurera alors de vérifier $\Delta_t^2\xi^8 << 1$ dans cet ensemble. Autrement dit, on imposera $$\Delta_t << \left(\frac{1}{\xi_\text{max}}\right)^4$$.
\paragraph{Remarque :} Le $\theta$ qui minimise le second terme de ce développement limité est $\theta = \frac{1}{2}$. Le schéma de Crank Nicholson est donc le meilleur ici aussi.

\subsubsection{Discrétisation temps et en espace :}
Discrétisons désormais aussi la variable d'espace. Posons $x_j = j\Delta_x$ pour tout $j\in\mathbb{Z}$ et intéressons nous à l'approximation $\texttt{U}^n_j \approx u(x_j,t^n)$, définie par le schéma suivant
\begin{equation}
\frac{\texttt{U}^{n+1}-\texttt{U}^n}{\Delta_t} + B(\theta\texttt{U}^{n+1} + (1-\theta)\texttt{U}^{n}) = 0 \label{num_scheme_Airy}
\end{equation}
où $\texttt{U}^{n+1}$ désigne cette fois la suite $(\texttt{U}^n_j)_{j\in\mathbb{Z}}$ et où B est un opérateur linéaire que nous déterminerons plus tard mais que nous supposerons antisymétrique.\\

\paragraph{Étude de la conservation de la norme $L^2$ de notre schéma}\,\\

Désignons par $<\cdot,\cdot>$ le produit scalaire associé à l'espace de Hilbert $l^2(\mathbb{Z})$ définit par $$<\texttt{u},\texttt{v} > = \sum\limits_{i\in\mathbb{Z}}\texttt{u}_i\texttt{v}_i\Delta_x$$ et observons l'évolution de $\texttt{U}^n$ pour la norme associée. Pour cela on effectue le produit scalaire des termes de l'équation (\ref{num_scheme_Airy}) par $ \texttt{V}^n_\theta = (\theta\texttt{U}^{n+1} + (1-\theta)\texttt{U}^{n})$. Ceci donne

\begin{equation}
    <\texttt{U}^{n+1}- \texttt{U}^n, \texttt{V}^n_\theta >+ \Delta_t < B\texttt{V}^n_\theta , \texttt{V}^n_\theta  > = 0 \label{num_scheme_L2_stability_eq_1}
\end{equation}
Comme $B$ est antisymétrique, il vient 
$$< B\texttt{V}^n_\theta , \texttt{V}^n_\theta  > = < \texttt{V}^n_\theta , B^T\texttt{V}^n_\theta  > = -< B\texttt{V}^n_\theta , \texttt{V}^n_\theta  > = 0$$
Et donc (\ref{num_scheme_L2_stability_eq_1}) devient

\begin{equation}\label{develop_theta_square}
\begin{split}
    0  &=   <\texttt{U}^{n+1}- \texttt{U}^n, \texttt{U}^{n} + \theta(\texttt{U}^{n+1}  - \texttt{U}^{n})> 
    \\ &=   <\texttt{U}^{n+1}- \texttt{U}^n, \texttt{U}^{n}>  + \theta\|\texttt{U}^{n+1}  - \texttt{U}^{n})\|^2
    \\ &=   <\texttt{U}^{n+1}, \texttt{U}^{n}>-  \|\texttt{U}^{n}\|^2 + \theta\|\texttt{U}^{n+1}  - \texttt{U}^{n}\|^2
    \\ &=   \frac{1}{2}\left(\|\texttt{U}^{n+1}\|^2+ \|\texttt{U}^n\|^2 - \|\texttt{U}^{n+1}- \texttt{U}^n\|^2\right)-  \|\texttt{U}^{n}\|^2 + \theta\|\texttt{U}^{n+1}  - \texttt{U}^{n}\|^2
    \\ &=   \frac{1}{2}\left(\|\texttt{U}^{n+1}\|^2 - \|\texttt{U}^n\|^2 - (1 - 2\theta) \|\texttt{U}^{n+1}-  \texttt{U}^{n}\|^2 \right)
\end{split}
\end{equation}

Autrement dit, 

\begin{equation}
    \|\texttt{U}^{n+1}\|^2 = \|\texttt{U}^n\|^2 + (1 - 2\theta) \|\texttt{U}^{n+1}-  \texttt{U}^{n}\|^2
\end{equation}

On en déduit que:\\

Si $\theta = \frac{1}{2}$, $\|\texttt{U}^{n+1}\|^2 = \|\texttt{U}^n\|^2$ et donc la norme $L^2$ numérique est conservée.\\

Si $\theta > \frac{1}{2}$, $\|\texttt{U}^{n+1}\|^2 < \|\texttt{U}^n\|^2$ et donc la norme $L^2$ décroît. (il y a dissipation d'énergie)\\

Si $\theta < \frac{1}{2}$, $\|\texttt{U}^{n+1}\|^2 > \|\texttt{U}^n\|^2$ et donc la norme $L^2$ croît. (le schéma est instable)\\

\paragraph{Choix d'un schéma pour la dérivée troisième}\,\\

Pour la suite nous poserons pour $B$ l'opérateur suivant
\begin{align*}
        B : l^2(\mathbb{Z}) &\longrightarrow l^2(\mathbb{Z})\\
        (\texttt{v}_i)_{i\in\mathbb{Z}} &\mapsto \left(\frac{\texttt{v}_{i-2} - 2\texttt{v}_{i-1} + 2\texttt{v}_{i+1}-\texttt{v}_{i+2} }{2(\Delta_x)^3}\right)_{i\in\mathbb{Z}}
\end{align*}

Cet opérateur est antisymétrique. Le schéma associé est consistant à l'ordre 2 à l'opérateur $ \partial_{xxx}$ .

\subsubsection{Implémentation numérique}

\begin{figure}[H]
    \centering
    \includegraphics[scale = 0.5]{graphs/v_xi en fonction de dt.pdf}
    \caption{Vitesse des harmoniques en fonction de la fréquence pour $\Delta_t = 3\times10^{-4}$}
    \label{fig:enter-label}
\end{figure}

\subsection{L'équation de KdV}
L'équation (\ref{KdV}) peut s'écrire

\begin{equation*}
    \partial_tu + \partial_{xxx}u +\frac{1}{2}\partial_x(u^2)  = 0
\end{equation*}
Lors de la simulation numérique de cette équation, nous nous assurerons que les schémas utilisés vérifient l'équivalent, au sens discret, de la proposition suivante:

\begin{prop}
    Soit $u$ une solution de (\ref{KdV}). Supposons que $u$ et toutes ses dérivées en espace tendent vers 0 lorsque $|x|$ tend vers l'infini (conformément à l'hypothèse ($\textbf{H}_9$) ). Supposons aussi que $u$ est $L_1$. Alors les deux égalités suivantes sont vérifiées.
    \begin{equation*}
        \partial_t\left(\int_{-\infty}^{+\infty}u\right) = 0
    \end{equation*}
    
    et 
    \begin{equation*}
        \partial_t\left(\int_{-\infty}^{+\infty}u^2\right) = 0
    \end{equation*}
\end{prop}
\begin{proof}
    Commençons par la moyenne de $u$:
    \begin{equation}
        \begin{split}
            \partial_t\left(\int_{-\infty}^{+\infty}u\right) &= \int_{-\infty}^{+\infty}\partial_tu\\
            &= \int_{-\infty}^{+\infty}\left(-\partial_{xxx}u -\frac{1}{2}\partial_x(u^2)\right)\\
            &= -\int_{-\infty}^{+\infty}\partial_{xxx}u -\frac{1}{2}\int_{-\infty}^{+\infty}\partial_x(u^2)\\
            &= -\left[\partial_{xx}u\right]_{-\infty}^{+\infty} -\frac{1}{2}\left[u^2\right]_{-\infty}^{+\infty}\\
        \end{split}
    \end{equation}
    
    Comme $\partial_xu$ et $u^2$ sont nuls en l'infini, on a donc l'égalité voulue.\\

    Regardons ensuite la norme $L^2$ de $u$:

    \begin{equation}
        \begin{split}
            \partial_t\left(\int_{-\infty}^{+\infty}u^2\right) &= \int_{-\infty}^{+\infty}\partial_t(u)^2\\
            &= \int_{-\infty}^{+\infty}2u\partial_t(u)\\
            &= 2\int_{-\infty}^{+\infty}u\times\left(-\partial_{xxx}u -\partial_x(u^2)\right)\\
            &= -2\int_{-\infty}^{+\infty}u\partial_{xxx}u -\int_{-\infty}^{+\infty}u\partial_x(u^2)\\
        \end{split}
    \end{equation}
    
    Après une intégration par partie, il vient alors
    \begin{equation}
        \begin{split}
            \partial_t\left(\int_{-\infty}^{+\infty}u^2\right)
            &= -2\left[u\partial_{xx}u\right]_{-\infty}^{+\infty} + 2\int_{-\infty}^{+\infty}\partial_xu\partial_{xx}u -\left[u^3\right]_{-\infty}^{+\infty} +\int_{-\infty}^{+\infty}u^2\partial_xu\\
            &= -2\left[u\partial_{xx}u\right]_{-\infty}^{+\infty} + 2\left[\frac{(\partial_{x}u)^2}{2}\right]_{-\infty}^{+\infty} -\left[u^3\right]_{-\infty}^{+\infty} + \left[\frac{u^3}{3}\right]_{-\infty}^{+\infty}\\
        \end{split}
    \end{equation}

    Comme toutes ces fonctions sont nulles en l'infini, tous les termes sont nuls et on obtient l'égalité voulue.
\end{proof}

\paragraph{Remarque: } Dans notre cas précis, on étudiera des solution sous la forme de vagues localisées telles que des solitons, on supposera alors une convergence très rapide vers 0 de la hauteur du fluide à mesure que l'on s'éloigne de cette vague. Dans ce contexte, il est alors sensé de supposer $u$ intégrable.

\subsubsection{ Discrétisation en temps: }

De manière analogue à (\ref{num_theta_scheme_Airy}), posons $t^n = n\Delta_t$, et intéressons à l'approximation $\texttt{U}^n \approx u(\cdot,t^n)$ générée par le $\theta-$schéma si dessous.

\begin{equation}
\frac{\texttt{U}^{n+1}-\texttt{U}^n}{\Delta_t} + \partial_{xxx}\left(\frac{\theta\texttt{U}^{n+1} + (1-\theta)\texttt{U}^{n}}{2}\right) + \frac{1}{2}\partial_{x}\left( \left(\frac{\theta\texttt{U}^{n+1} + (1-\theta)\texttt{U}^{n}}{2}\right)^2 \right) = 0 \label{num_theta_scheme_kdv}
\end{equation}
On supposera que les fonctions $\texttt{U}^n$ et toutes leurs dérivées sont nulles à l'infini ( conformément à l'hypothèse ($\textbf{H}_9$) ).
\paragraph{Étude de l'évolution de la moyenne de $\texttt{U}^n$}\,\\

Notons par $\overline{\texttt{U}}^n$ la moyenne $\int_{-\infty}^{+\infty}\texttt{U}^n$. En intégrant l'équation (\ref{num_theta_scheme_kdv}) on trouve la relation suivante:
\begin{equation*}
\begin{split}
    \frac{\overline{\texttt{U}}^{n+1}-\overline{\texttt{U}}^n}{\Delta_t} 
    &= -\int_{-\infty}^{+\infty} \partial_{xxx}\left(\frac{\theta\texttt{U}^{n+1} + (1-\theta)\texttt{U}^{n}}{2}\right) -\frac{1}{2}\int_{-\infty}^{+\infty} \partial_{x}\left( \left(\frac{\theta\texttt{U}^{n+1} + (1-\theta)\texttt{U}^{n}}{2}\right)^2 \right) \\
    &= - \left[ \partial_{xx}\left(\frac{\theta\texttt{U}^{n+1} + (1-\theta)\texttt{U}^{n}}{2}\right)\right]_{-\infty}^{+\infty} -\frac{1}{2}\left[\left(\frac{\theta\texttt{U}^{n+1} + (1-\theta)\texttt{U}^{n}}{2}\right)^2 \right]_{-\infty}^{+\infty} \\
    & = 0
\end{split}
\end{equation*}

Ainsi, pour tout entier naturel $n$, on a $\overline{\texttt{U}}^n = \overline{\texttt{U}}^0$.

\paragraph{Étude de l'évolution la norme $\textbf{L}^2$ de $\texttt{U}^n$}\,\\

Posons $\texttt{V}_\theta^n = \frac{\theta\texttt{U}^{n+1} + (1-\theta)\texttt{U}^{n}}{2}$. En multipliant (\ref{num_theta_scheme_kdv}) par $\texttt{V}^n_\theta$ puis en intégrant, on trouve la relation suivante:
\begin{equation*}
\begin{split}
    \int_{-\infty}^{+\infty}\frac{\texttt{U}^{n+1}-\texttt{U}^n}{\Delta_t}\texttt{V}_\theta^n
    &= -\int_{-\infty}^{+\infty} \texttt{V}_\theta^n\partial_{xxx} \texttt{V}_\theta^n  -\frac{1}{2}\int_{-\infty}^{+\infty} \texttt{V}_\theta^n\partial_{x}\left( \left(\texttt{V}_\theta^n\right)^2 \right) \\
    (i.p.p)~~&= \int_{-\infty}^{+\infty} \partial_{x}\texttt{V}_\theta^n\partial_{xx} \texttt{V}_\theta^n  + \frac{1}{2}\int_{-\infty}^{+\infty}  \left(\texttt{V}_\theta^n\right)^2\partial_{x} \texttt{V}^n  \\
    &= \frac{1}{2}\left[\left(\partial_{x}\texttt{V}_\theta^n\right)^2\right]_{-\infty}^{+\infty} -\frac{1}{6} \left[\left(\texttt{V}_\theta^n\right)^3 \right]_{-\infty}^{+\infty} \\
    & = 0 .
\end{split}
\end{equation*}

Remarquons de plus, par un développement analogue à (\ref{develop_theta_square}), que 
$$(\texttt{U}^{n+1}-\texttt{U}^n)\texttt{V}_\theta^n = \frac{1}{2}\left((\texttt{U}^{n+1})^2-(\texttt{U}^n)^2 + (1-2\theta)(\texttt{U}^{n+1}-\texttt{U}^{n})^2 \right).$$ 

On obtient alors la relation
\begin{equation}
    \int_{-\infty}^{+\infty}(\texttt{U}^{n+1})^2 - \int_{-\infty}^{+\infty}(\texttt{U}^{n})^2 = (1-2\theta) \int_{-\infty}^{+\infty}(\texttt{U}^{n+1}- \texttt{U}^{n})^2 .
\end{equation}

On tire ici les même conclusion que pour l'équation de Airy, c'est à dire:\\

Si $\theta = \frac{1}{2}$, la norme $L^2$ numérique est conservée.\\

Si $\theta > \frac{1}{2}$, la norme $L^2$ décroît. (il y a dissipation d'énergie)\\

Si $\theta < \frac{1}{2}$, la norme $L^2$ croît. (le schéma est instable)\\




\subsubsection{Schéma numérique}



De manière analogue à (\ref{num_scheme_Airy}), posons $x_j = j\Delta_x$ et $t^n = n\Delta_t$  et intéressons nous à l'approximation $\texttt{U}^n_j \approx u(x_j,t^n)$ définie par le schéma aux différences finies suivant
\begin{equation}
\frac{\texttt{U}^{n+1}-\texttt{U}^n}{\Delta_t} + B\left(\frac{\texttt{U}^{n+1} + \texttt{U}^{n}}{2}\right) + \frac{1}{2}D\left( \left(\frac{\texttt{U}^{n+1} + \texttt{U}^{n}}{2}\right)^2 \right) = 0 \label{num_scheme_kdv}
\end{equation}
Où $B$ et $D$ sont deux opérateurs antisymétriques. Pour tout entier $n$, on supposera que la suite $(U^n_j)_{j\in\mathbb{Z}}$ est sommable (i.e. qu'elle est dans $l^1(\mathbb{Z})$). Par abus de notation, on notera $<v,\textbf{1}>$ la somme $\sum_{j\in\mathbb{Z}}v_j$, où \textbf{1} désigne la suite constante égale à 1, indexée par $\mathbb{Z}$ et où $<\cdot,\cdot>$ désigne le produit scalaire dans $l^2(\mathbb{Z})$ précédemment défini.

\paragraph{Évolution moyenne de $\texttt{U}^n$}\,\\

Les opérateurs $B$ et $D$ correspondant à une approximation aux sens des différences finies de $\partial_{xxx}$ et $\partial_x$ respectivement, on à les égalités $B\textbf{1} = D\textbf{1} = 0 $. On s'intéresse à l'évolution de la somme $<\texttt{U}^n,\textbf{1}> := \sum_j \texttt{U}_j^n$ lorsque $n$ croît. En sommant sur $j\in\mathbb{Z}$ les coefficients des suites présentes dans l'équation (\ref{num_scheme_kdv}), on obtient la relation suivante:

\begin{equation*}
    \frac{<\texttt{U}^{n+1},\textbf{1},>-<\texttt{U}^n,\textbf{1}>}{\Delta_t} + <B\left(\frac{\texttt{U}^{n+1} + \texttt{U}^{n}}{2}\right),\textbf{1}> + \frac{1}{2}<D\left( \left(\frac{\texttt{U}^{n+1} + \texttt{U}^{n}}{2}\right)^2 \right),\textbf{1}> = 0 .
\end{equation*}

Par antisymétrie de $B$, 

\begin{equation*}
    \begin{split}
        <B\left(\frac{\texttt{U}^{n+1} + \texttt{U}^{n}}{2}\right),\textbf{1}> &= -<\left(\frac{\texttt{U}^{n+1} + \texttt{U}^{n}}{2}\right),B\textbf{1}>\\
        &= -<\left(\frac{\texttt{U}^{n+1} + \texttt{U}^{n}}{2}\right),0>\\
        &= 0.
    \end{split}
\end{equation*}

De même, par antisymétrie de $D$, 
\begin{equation*}
    \begin{split}
        <D\left( \left(\frac{\texttt{U}^{n+1} + \texttt{U}^{n}}{2}\right)^2 \right),\textbf{1}> &= 0.
    \end{split}
\end{equation*}

Il vient alors 
\begin{equation*}
    \frac{<\texttt{U}^{n+1},\textbf{1},>-<\texttt{U}^n,\textbf{1}>}{\Delta_t} = 0 .
\end{equation*}

Ainsi, pour tout entier naturel $n$, on a $<\texttt{U}^n,\textbf{1}> = <\texttt{U}^0,\textbf{1}>$.

\paragraph{Évolution norme $l^2$ de $\texttt{U}^n$ dans le cas d'un schéma d'ordre 2 de la dérivée pour l'opérateur $D$}\,\\

Pour cette partie, posons comme $D$ l'opérateur suivant:
\begin{align*}
        D : l^2(\mathbb{Z}) &\longrightarrow l^2(\mathbb{Z})\\
        (\texttt{v}_j)_{j\in\mathbb{Z}} &\mapsto \left(\frac{\texttt{v}_{j+1} - \texttt{v}_{j-1} }{2\Delta_x}\right)_{j\in\mathbb{Z}}
\end{align*}
On souhaite étudier l'évolution de la norme $l^2$ de $\texttt{U}^n$.  Pour cela on effectue le produit scalaire des termes de l'équation (\ref{num_scheme_kdv}) par $ \texttt{V}^n = \frac{(\texttt{U}^{n+1} + \texttt{U}^{n})}{2} = (v^n_j)_{j\in\mathbb{Z}}$. Ceci donne

\begin{equation*}
    \frac{<\texttt{U}^{n+1} - \texttt{U}^n,\texttt{V}^n>}{\Delta_t} + <B\texttt{V}^n,\texttt{V}^n> + \frac{1}{2}<D (\texttt{V}^n)^2 ,\texttt{V}^n> = 0 .
\end{equation*}

Remarquons que $<\texttt{U}^{n+1} - \texttt{U}^n,\texttt{V}^n> = \|\texttt{U}^{n+1}\|^2-\|\texttt{U}^{n}\|^2$. De plus, par antisymétrie de $B$, on sait que $<B\texttt{V}^n,\texttt{V}^n> = 0$. Ces simplifications nous amènent à l'égalité suivante
\begin{equation*}
    \frac{\|\texttt{U}^{n+1}\|^2-\|\texttt{U}^{n}\|^2}{\Delta_t} = -\frac{1}{2}<D (\texttt{V}^n)^2 ,\texttt{V}^n> .
\end{equation*}

Dès lors, prenons compte de la proposition suivante:
\begin{prop}
    Soit $\textbf{V} = (v_j)_{j\in\mathbb{V}}$ une suite appartenant à $l^2(\mathbb{Z})$ telle que $\textbf{V}^2:=(v_j^2)_{j\in\mathbb{V}}$ est aussi dans $l^2(\mathbb{Z})$ et soit $D$ l'opérateur précédemment explicité. Alors l'égalité suivante est vérifiée
    \begin{equation}
        <D \textbf{V}^2 ,\textbf{V}> = -\frac{(\Delta_x)^2}{6}<(\tilde{D}\textbf{V})^3,\textbf{1}>
    \end{equation}
    Où $\tilde{D}$ est définie par $(\tilde{D}\textbf{V})_j = \frac{v_{j+1}-v_j}{\Delta_x}$ et représente une approximation à l'ordre $\mathcal{O}(\Delta_x)$ de l'opérateur $\partial_x$.
\end{prop}
\begin{proof}
    
    Par antisymétrie de $D$, 
    $$<D \textbf{V}^2 ,\textbf{V}> = -< \textbf{V}^2 ,D\textbf{V}>$$

    En développant le terme de droite, on a donc
    \begin{equation*}
        \begin{split}
            <D \textbf{V}^2 ,\textbf{V}> &= -\frac{1}{2\Delta_x}\sum\limits_{j\in\mathbb{Z}}v_j^2(v_{j+1}-v_{j-1})\\
            &= -\frac{1}{2\Delta_x}\sum\limits_{j\in\mathbb{Z}}v_j^2v_{j+1}-v_j^2v_{j-1}\\
            &= -\frac{1}{2\Delta_x}\left(\sum\limits_{j\in\mathbb{Z}}v_j^2v_{j+1} - \sum\limits_{j\in\mathbb{Z}}v_j^2v_{j-1}\right)
        \end{split}
    \end{equation*}

    Si on ré-indexe la somme de droite par le changement de variable $j'  = j-1$, on obtient alors
    \begin{equation*}
        \begin{split}
            <D \textbf{V}^2 ,\textbf{V}> &= -\frac{1}{2\Delta_x}\left(\sum\limits_{j\in\mathbb{Z}}v_j^2v_{j+1} - \sum\limits_{j\in\mathbb{Z}}v_{j+1}^2v_{j}\right)\\
            &= -\frac{1}{2\Delta_x}\sum\limits_{j\in\mathbb{Z}}(v_j^2v_{j+1} - v_{j+1}^2v_{j})\\
            &= -\frac{1}{2\Delta_x}\frac{1}{3}\sum\limits_{j\in\mathbb{Z}}(3v_j^2v_{j+1} - 3v_{j+1}^2v_{j})\\
        \end{split}
    \end{equation*}

    En partant de la formule du binôme de Newton  $(a-b)^3 = a^3 - 3a^2b + 3ab^2 - b^3$ pour $a = v_{j+1}$ et $b = v_j$, il vient que 
    $$v_j^2v_{j+1} - v_{j+1}^2v_{j} = \frac{1}{3}\left( (v_{j+1} - v_j)^3 - v_{j+1}^3 + v_j^3\right)$$

    Et donc
    \begin{equation*}
        \begin{split}
            <D \textbf{V}^2 ,\textbf{V}> &= -\frac{1}{6\Delta_x}\left(\sum\limits_{j\in\mathbb{Z}}(v_{j+1} - v_j)^3 - \sum\limits_{j\in\mathbb{Z}}(v_{j+1}^3 - v_j^3)\right)\\
        \end{split}
    \end{equation*}

    Les termes de la somme de droite se télescopent, et $v_j$ tend vers 0 en $+\infty$ et en $-\infty$. Cette somme est donc nulle. Il nous reste donc
    $$<D \textbf{V}^2 ,\textbf{V}> = -\frac{1}{6\Delta_x}\sum\limits_{j\in\mathbb{Z}}(v_{j+1} - v_j)^3$$
    
    Autrement dit,
    $$<D \textbf{V}^2 ,\textbf{V}> = -\frac{(\Delta_x)^2}{6}\sum\limits_{j\in\mathbb{Z}}\left(\frac{v_{j+1} - v_j}{\Delta_x}\right)^3$$
\end{proof}

On est finalement ammené à la relation 
\begin{equation*}
    \frac{\|\texttt{U}^{n+1}\|^2-\|\texttt{U}^{n}\|^2}{\Delta_t} = (\Delta_x)^2\frac{1}{12}<(\Tilde{D}\texttt{V}^n)^3 ,\texttt{1}> .
\end{equation*}

Ainsi, en assumant une certaine hypothèse de modération de la pente de $u$, ce qui est envisageable si on étudie de faibles longueurs d'onde, le terme  $(\Delta_x)^2\frac{1}{12}<(\Tilde{D}\texttt{V}^n)^3 ,\texttt{1}> $ peut être considéré comme égal à $\mathcal{O}((\Delta_x)^2)$. Ce qui nous amène à 
\begin{equation*}
    \frac{\|\texttt{U}^{n+1}\|^2-\|\texttt{U}^{n}\|^2}{\Delta_t} = \mathcal{O}((\Delta_x)^2).
\end{equation*}

\subsubsection{Les solitons de KdV}
On cherche des solutions de l'équation de (\ref{KdV}) sous la forme d'ondes voyageuses, qu'on appellera solitons. 
\begin{prop}
    Soit $c \in \mathbb{R}$ et soit $g:\mathbb{R}\rightarrow\mathbb{R}$ telle que $ (x,t)\mapsto g(x - ct)$ est solution de (\ref{KdV}). Alors $g$ vérifie
    \begin{equation}
     (g')^2  = cg^2 - \frac{1}{3}g^3 \label{eq_soliton}
    \end{equation}\\
\end{prop}
\begin{proof}
    En injectant g dans (\ref{KdV}) on en déduit une équation régissant $g$.
\begin{equation*}
    -cg' + g^{(3)} + gg' = 0 
\end{equation*}

Cette dernière peut aussi s'écrire
\begin{equation*}
    \left(-cg + g'' + \frac{1}{2}g^2\right)' = 0 
\end{equation*}

En intégrant, il vient alors que
\begin{equation*}
        -cg + g'' + \frac{1}{2}g^2 = k_1
\end{equation*}

Sachant l'hypothèse $(\textbf{H}_9)$, on sait que $k_1$ = 0 . Ce qui nous amène à 
\begin{equation*}
        -cg + g'' + \frac{1}{2}g^2 = 0
\end{equation*}

En multipliant cette équation par $g'$, il vient 
\begin{equation*}
        \left(-\frac{c}{2}g^2 + \frac{1}{2}(g')^2 + \frac{1}{6}g^3\right)' = 0
\end{equation*}

En intégrant, et en considérant à nouveau l'hypothèse $(\textbf{H}_9)$, on obtient
\begin{equation*}
     -cg^2 + (g')^2 + \frac{1}{3}g^3 = 0
\end{equation*}
\end{proof}

\cite{Garnier} et \cite{MT} proposent une astuce pour trouver une solution à cette équation. Elle revient à considérer la fonction transcendante élémentaire $\text{ch}(x) = \frac{e^x+ e^{-x}}{2}$ et de chercher $g$ sous la forme 

\begin{equation*}
     g(x) = \lambda\text{ch}^{\omega}(\kappa x)
\end{equation*}

Sachant que $(\text{ch}')^2 = \text{ch}^2 - 1$, on trouve alors
\begin{list}{}{}
    \item[\textbullet]~~ $(g')^2 = \lambda^2\omega^2\kappa^2\left( \text{ch}^{2\omega}(\kappa x) - \text{ch}^{2\omega-2}(\kappa x)\right)$
    \item[\textbullet]~~ $cg^2 = c \lambda^2\text{ch}^{2\omega}(\kappa x)$
    \item[\textbullet]~~ $\frac{1}{3}g^3 = \frac{\lambda^3}{3}\text{ch}^{3\omega}(\kappa x)$
    
\end{list}\,\\

On cherche alors $\lambda$, $\omega$ et $\kappa$ tels que

\begin{equation*}
    \lambda^2\omega^2\kappa^2 \text{ch}^{2\omega}(\kappa x) - \lambda^2\omega^2\kappa^2 \text{ch}^{2\omega-2}(\kappa x) - c \lambda^2\text{ch}^{2\omega}(\kappa x) + \frac{\lambda^3}{3}\text{ch}^{3\omega}(\kappa x) = 0
\end{equation*}

On remarque qu'il suffit alors que 
\begin{align*}
    & 2\omega - 2 = 3\omega\\
    &\lambda^2\omega^2\kappa^2= c\lambda^2 \\ 
    & \lambda^2\omega^2\kappa^2 = \frac{\lambda^3}{3}
\end{align*}

Ce qui équivaut à
\begin{align*}
    &\omega = -2\\
    &\kappa = \frac{\sqrt{c}}{2}\\
    &\lambda = 3c
\end{align*}

En résumé, une solution au problème (\ref{eq_soliton}) est

\begin{equation}
    u(x,t) = \frac{3c}{\text{ch}^2\left( \frac{\sqrt{c}}{2}(x-ct)\right)}
\end{equation}

\begin{figure}[H]
\centering
\begin{subfigure}
  \centering
  \includegraphics[width=0.45\linewidth]{graphs/Soliton_t0.pdf}
  \label{fig:sub1}
\end{subfigure}%
\begin{subfigure}
  \centering
  \includegraphics[width=0.45\linewidth]{graphs/Soliton_t1.pdf}
  \label{fig:sub2}
\end{subfigure}
\caption{Simulation du soliton pour c = 1.} 
\end{figure}


\newpage
\nocite{MT}
\nocite{Lannes}
\printbibliography
\end{document}
\paragraph{Exemple :} Après ces changements de variables, (\ref{eq_B2}) devient $$\partial_{z'z'}\phi' + \mu \partial_{x_1' x_1'}\phi'  = 0 ~~~ \text{si }\textbf{X} = \mathbb{R} $$
$$\partial_{z'z'}\phi' + \mu \partial_{x_1'x_1'}\phi' + \mu \gamma^2 \partial_{x_2'x_2'}\phi' = 0 ~~~ \text{si }\textbf{X} = \mathbb{R}^2$$