\documentclass[12pt,a4paper]{article}
\usepackage[utf8]{inputenc}
\usepackage[francais]{babel}
\usepackage{amsmath}
\usepackage{amsfonts}
\usepackage{amssymb}
\usepackage{stmaryrd}
\usepackage{float}
\usepackage{systeme}
\setlength{\oddsidemargin}{-18pt}
\setlength{\evensidemargin}{-18pt}
\setlength{\textwidth}{481pt}
\setlength{\textheight}{710pt}
\setlength{\headheight}{-53pt}
\setlength{\footskip}{18pt}
\usepackage[hyperindex]{hyperref}
\newtheorem{dfn}{\textbf{Définition}}[subsection]
\newtheorem{thm}[dfn]{\textbf{Théoreme}}
\newtheorem{prop}[dfn]{\textbf{Proposition}}
\newtheorem{cor}[dfn]{\textbf{Corollaire}}
\newtheorem{lem}[dfn]{\textbf{Lemme}}
\usepackage[nothm]{thmbox}

\newcommand{\rot}{\text{rot}}

\numberwithin{equation}{section}
\def\proof{\begin{thmbox}[M]{\textbf{Preuve :}}}
\def\endproof{\end{thmbox}}
\begin{document}
\title{Analyse spectrale d'écoulements fluides et modèles d'amortissement}
\author{Lesage Adrien}
\maketitle
\newpage
\tableofcontents
\newpage
\section{Introduction aux équations d'Euler à surface libre.}
\subsection{Hypothèses}

Tout le long de ce mémoire, nous nous intéresserons à l'étude d'un fluide dont nous supposerons les hypothèses suivantes vérifiées
\begin{list}{}{}
    \item[$(\textbf{H}_1)$ : ] Le fluide est \textit{homogène} et \textit{incompressible}.
    \item[$(\textbf{H}_2)$ : ] Le fluide est \textit{non visqueux}.
    \item[$(\textbf{H}_3)$ : ] Le fluide est \textit{irrotationnel}.
\end{list}

Soit $\textbf{X} = \mathbb{R}^d$ avec $d \in \{1,2\}$. On souhaite étudier un tel fluide évoluant dans le domaine $$\Omega_t = \left\{ (x,z) \in \textbf{X}\times\mathbb{R}  \,|\quad b(x) - H_0 \leq z \leq \zeta(t,x) \,\right\}$$
\begin{list}{}{}
\item[\textbullet] $b : \textbf{X}\rightarrow \mathbb{R}$ désigne les variations du fond du fluide, de profondeur caractéristique $H_0 > 0$
\item[\textbullet] le graphe $\zeta : \mathbb{R}_+\times \textbf{X}$ désigne la surface du fluide, de sorte que les hypothèses suivante soient vérifiées
\end{list}
\begin{list}{}{}
    \item[$(\textbf{H}_4)$ : ] Les particules de fluide ne traversent pas le fond.
    \item[$(\textbf{H}_5)$ : ] Les particules de fluide ne traversent pas la surface.
    \item[$(\textbf{H}_6)$ : ]  En tout point $x\in\textbf{X}$ et en tout instant $t\geq 0$, l'épaisseur du fluide $\zeta(t,x) - (b(x)-H_0)$ est supérieure à une constante $H_0>0$ indépendante de $x$ et de $t$.
    \item[$(\textbf{H}_7)$ : ] La pression  du fluide à la surface est égale à la pression atmosphérique $P_{\text{atm}}$ (on néglige les tensions de surface).
    \item[$(\textbf{H}_8)$ : ] Le fluide est soumis à une force de gravité d'intensité $g$ et de direction opposée au vecteur $\textbf{z} = (0_{\textbf{X}},1)$.
    \item[$(\textbf{H}_9)$ : ] Le fluide est au repos à l'infini et $\lim\limits_{\|x\|\rightarrow \infty}\zeta(x,t) = 0$
\end{list}

\begin{center}
    \texttt{ajouter schéma}
\end{center}


\subsection{Formulation des équations d'Euler.}
\subsubsection{Formulation de l'évolution du champ de vitesse}
Pour commencer, nous allons adopter un point de vue lagrangien:
\begin{list}{}{}
    \item[\textbullet] Posons $\left(\begin{array}{l}
         \mathcal{X}\\
         \mathcal{Z}
    \end{array}\right) :  \mathbb{R}_+ \rightarrow \textbf{X}\times\mathbb{R}$ la fonction qui suit le déplacement d'une particule de fluide de position initiale $\left(\begin{array}{l}
         \mathcal{X}\\
         \mathcal{Z}
    \end{array}\right)(0)  \in \Omega_0$.
    \item[\textbullet] Posons $\textbf{U}(t,x,z)$ la vitesse d'une particule de fluide située en $(x,z) \in \Omega_t$ à l'instant $t$.
\end{list}
Les hypothèses $(\textbf{H}_4)$ et $(\textbf{H}_5)$ nous donnent que $\left(\begin{array}{l}
         \mathcal{X}\\
         \mathcal{Z}
\end{array}\right)\in \Omega_t$ pour tout instant $t \geq 0$. Ceci nous permet d'écrire
    
\begin{equation}
\frac{d}{dt} \left(\begin{array}{l}
         \mathcal{X}\\
         \mathcal{Z}
\end{array}\right) = U(t,\mathcal{X}(t), \mathcal{Z}(t)) 
\end{equation}

En dérivant une fois de plus, on obtient 
$$
\frac{d^2 }{dt^2}\left(\begin{array}{l}
         \mathcal{X}\\
         \mathcal{Z}
\end{array}\right)  = \frac{\partial \textbf{U}(t,\mathcal{X}(t),\mathcal{Z}(t))}{\partial t} + \text{Jac}(\textbf{U})\frac{d}{d t}\left(\begin{array}{l}
         \mathcal{X}\\
         \mathcal{Z}
\end{array}\right) 
$$
    Ou, autrement dit,
    
$$
\frac{d^2 }{dt^2}\left(\begin{array}{l}
         \mathcal{X}\\
         \mathcal{Z}
\end{array}\right)= \left(\frac{\partial\textbf{ U}}{\partial t}+ (\textbf{U}\cdot \nabla ) \textbf{U}\right)(t,\mathcal{X}(t),\mathcal{Z}(t)) 
$$

Les hypothèses $(\textbf{H}_2)$ nous dit que les particules ne s'influencent pas mutuellement en dehors de la pression. Chaque particule n'est alors soumis qu'à deux forces: la force de gravité $ F_\text{g} = -\rho g\textbf{z}$ et la force de pression $F_\text{P} = - \nabla\textbf{P}$. La deuxième loi de Newton donne alors

$$
\rho\frac{d^2 }{dt^2}\left(\begin{array}{l}
         \mathcal{X}\\
         \mathcal{Z}
\end{array}\right)(t) = - \rho g \textbf{z} -\nabla P(t,X(t),\mathcal{Z}(t))
$$
Ce qui nous donne les équations d'Euler définies par le système suivant.

\begin{equation} 
\tag{$\textbf{E}_1$} \label{eq_E1}
\rho \left( \frac{\partial U}{\partial t} + (\textbf{U}\cdot \nabla ) \textbf{U} \right) + \rho g\textbf{z} + \nabla \textbf{P} = 0
\end{equation}


\subsubsection{Formulation de la condition d'incompressibilité}
L'hypothèse $(\textbf{H}_1)$ se traduit par
$$\rho(t,x,z) = \rho_0$$
En conséquence, la loi de conservation de la masse d'un fluide
\begin{equation} \label{mass_conservation}
    \frac{\partial \rho}{\partial t} - \text{div}(\rho\textbf{U}) = 0
\end{equation}
devient
\begin{equation} \tag{$\textbf{E}_2$} \label{divergence_free}
    \text{div} (\textbf{U}) = 0
\end{equation}

\subsubsection{Formulation des conditions aux bords et de l'évolution de la surface}

L'hypothèse $(\textbf{H}_4)$ s'exprime ainsi

\begin{equation}
    \tag{${\textbf{E}}_3$} \label{eq_E2}
    \textbf{U}.\textbf{n} = 0 \text{  sur }\{z = b(x)-H_0\}
\end{equation} 

où \textbf{n} désigne le vecteur normal extérieur au domaine $\Omega_t$.\\

Cette formule est une réécriture immédiate $(\textbf{H}_4)$  car $b-H_0$ ne dépend pas du temps. En revanche, comme $\zeta$ dépend du temps, la caractérisation de l'hypothèse $(\textbf{H}_5)$ est moins immédiate et est l'objet de la proposition suivante.



\begin{prop} L'hypothèse $(\textbf{H}_5)$ nous donne que
    \begin{equation} \tag{$\textbf{E}_4$} \label{eq_E3}
        \frac{\partial \zeta}{\partial t}(t,x) - \sqrt{1+ \| \nabla_\textbf{X}\zeta\|^2} \textbf{U} \cdot \textbf{n} = 0 \quad\text{dans}\,\{z = \zeta(t,x)\}
    \end{equation}
    où $\textbf{n}$ désigne le vecteur normal extérieur au domaine $\Omega_t$
\end{prop}
\begin{proof}
\textit{Etape 1 : }Commençons par déterminer $\textbf{n}$ en fonction de $t$, $x$, et $\zeta$. A $t$ fixé, la surface est paramétrée par l'application $$S_t: x\in \textbf{X} \mapsto \left(\begin{array}{l}
         x\\
         \zeta(t,x)
    \end{array}\right).$$
Sa différentielle en $x$ est 
$$DS_t(x) : v \in \textbf{X} \mapsto DS_t(x).v = \left(\begin{array}{l}
         v\\
         \nabla_X\zeta(t,x) \cdot v
    \end{array}\right).$$
Le vecteur normal \textbf{n} vérifie alors $$\textbf{n}\cdot (DS_t(x).v) = 0$$
pour tout $v \in \textbf{X}$. En posant $\textbf{n} = \left(\begin{array}{l}
         n_1\\
         n_2
    \end{array}\right)$, on trouve alors
    $$ \left(n_1 + n_2\nabla_X\zeta(t,x)\right)\cdot v = 0$$
pour tout $v\in\textbf{X}$. On a alors
$$ n_1 =  - n_2\nabla_X\zeta(t,x)$$
Donc $\textbf{n} = n_2\left(\begin{array}{l}
         - \nabla_X\zeta(t,x)\\
         1
    \end{array}\right)$
Comme la normale est dirigée vers le haut, il vient que $n_2>0$ et comme $\|\textbf{n}\| = 1$, on trouve finalement \begin{equation} \label{normal_surface_vector}
\textbf{n} = \frac{1}{\sqrt{1+\|\nabla_X\zeta(t,x)\|^2}}\left(\begin{array}{l}
         - \nabla_X\zeta(t,x)\\
         1
    \end{array}\right)\end{equation}
\textit{Etape 2 : }Considérons à nouveau $\left(\begin{array}{l}
         \mathcal{X}\\
         \mathcal{Z}
\end{array}\right)$ 
la trajectoire d'une particule et supposons qu'elle est initialement située à la surface. l'hypothèse nous donne qu'alors la particule ne peut traverser la surface ni dans un sens, ni dans l'autre. Il y a alors, pour tout temps $t \geq 0$, l'égalité suivante.
$$\mathcal{Z} - \zeta( . ,\mathcal{X}) = 0 $$
En dérivant cette égalité, il vient
\begin{align*} 
    0 &= \frac{d\mathcal{Z}}{dt} - \frac{\partial \zeta}{\partial t}(.,\mathcal{X}) - \nabla_\textbf{X}\zeta(.,\mathcal{X}) \cdot \frac{d \mathcal{X}}{dt} \\ 
    &=  \textbf{U}\cdot  \left(\begin{array}{l}
         - \nabla_X\zeta(t,x)\\
         1
    \end{array}\right) - \frac{\partial \zeta}{\partial t}(.,\mathcal{X}) 
    \\&= \sqrt{1 + \|\zeta\|^2}\textbf{U}  \cdot \textbf{n} - \frac{\partial \zeta}{\partial t}(.,\mathcal{X}) 
\end{align*}



\end{proof}

\paragraph{Pour résumer,} en combinant (\ref{eq_E1}), (\ref{divergence_free}), (\ref{eq_E2}) et (\ref{eq_E3}), on obtient les équations d'Euler à surface libre
\begin{equation} \tag{\textbf{E}}
    \left\{ 
    \begin{aligned}
        &\rho \left ( \frac{\partial \textbf{U}}{\partial t} + (\textbf{U}\cdot \nabla ) \textbf{U}+  g\textbf{z}  \right) + \nabla \textbf{P} = 0 & &  
        \\
        &\text{div}(\textbf{U}) = 0 & &  
        \\
        &\textbf{U}.\textbf{n} = 0 &\text{dans}\,\{z = b(x)-H_0\}& 
        \\
        &\frac{\partial \zeta}{\partial t}(t,x) - \sqrt{1+ \| \nabla_\textbf{X}\zeta\|^2} \textbf{U} \cdot \textbf{n} = 0 &\text{dans}\,\{z = \zeta(t,x)\} & 
    \end{aligned}
    \right.
\end{equation}
\paragraph{Remarque :} L'égalité vectorielle suivante
$$(\textbf{U}\cdot \nabla ) \textbf{U} = \rot(\textbf{U})\wedge \textbf{U} + \frac{1}{2}\nabla |\textbf{U}|^2$$
Nous permet de réécrire (\ref{eq_E1}) sous la forme
\begin{equation} \label{eq_E1bis}
    \rho \left( \frac{\partial \textbf{U}}{\partial t} +\rot(\textbf{U})\wedge \textbf{U} + \frac{1}{2}\nabla |\textbf{U}|^2+  g\textbf{z}  \right) + \nabla \textbf{P} = 0 
\end{equation}

Ainsi, l'hypothèse $(\textbf{H}_3)$, qui se traduit par $\rot(\textbf{U})=0$, permet alors d'avoir 
\begin{equation} \tag{$\textbf{E}_{1,\text{bis}}$} \label{eq_E1bis}
        \rho \left( \frac{\partial \textbf{U}}{\partial t} +\frac{1}{2}\nabla  |\textbf{U}|^2 +  g\textbf{z}\right) + \nabla \textbf{P} = 0 
\end{equation}

\subsection{Formulation de Bernoulli}

Grace notament à $(\textbf{H}_6)$, on sait que $\Omega_t$ est simplement connexe à tout instant $t$, et comme $\rot(\textbf{U}) = 0$, on en déduit alors l'existence d'une fonction $\Phi(t,.,.): \textbf{X}\times \mathbb{R} \rightarrow \mathbb{R}$ telle que $$\textbf{U} = \nabla \phi  \;.$$    

Ceci permet de réécrire le système (\ref{eq_E1bis}).

\begin{align*}
    0 = &\rho \left ( \frac{\partial \nabla \phi }{\partial t} +\frac{1}{2}\nabla  |\nabla \phi|^2  +  g\textbf{z} \right)+ \nabla \textbf{P} 
    \\
    =&\nabla \left ( \rho \left ( \frac{\partial \phi }{\partial t} +\frac{1}{2}|\nabla \phi|^2 +  gz \right) + \textbf{P} \right)
    \\
    =&\nabla \left ( \rho \left ( \frac{\partial \phi }{\partial t} +\frac{1}{2}|\nabla \phi|^2 +  gz \right) + \textbf{P}  - \textbf{P}_{\text{atm}}\right)
\end{align*}

Une condition suffisante pour verifier (\ref{eq_E1bis}) est alors
\begin{equation} \tag{$\textbf{B}_1$} \label{eq_B1}
  \frac{\partial \phi }{\partial t} +\frac{1}{2}|\nabla \phi|^2 +  gz   = \frac{1}{\rho}\left( \textbf{P}_{\text{atm}}  - \textbf{P}\right)
\end{equation}

Et comme $\text{div}(\nabla \phi) = \Delta \phi$, la condition d'incompressibilité (\ref{divergence_free}) devient alors

\begin{equation} \tag{$\textbf{B}_2$} \label{eq_B2}
    \Delta \phi = 0
\end{equation}

L'équation (\ref{eq_E2}), devient

\begin{equation} \tag{$\textbf{B}_3$} \label{eq_B3}
    \frac{\partial \phi}{\partial \textbf{n}} = 0 ~~~ \text{dans}\,\{z = b(x)-H_0\} .
\end{equation}

L'équation (\ref{eq_E3}), devient

\begin{equation} \tag{$\textbf{B}_4$} \label{eq_B4}
\frac{\partial \zeta}{\partial t}  - \sqrt{1+ \| \nabla_\textbf{X}\zeta\|^2}\frac{\partial \phi}{\partial\textbf{n}} = 0~~~\text{dans}\,\{z = \zeta(t,x)\} .
\end{equation}

Sachant (\ref{normal_surface_vector}), on connaît et on utilisera souvent l'équivalence entre (\ref{eq_B4}) et l'équation suivante


\begin{equation} \label{eq_B4_bis}
    \frac{\partial \zeta}{\partial t}  + \nabla_\textbf{X}\zeta\cdot\nabla_\textbf{X}\phi - \frac{\partial\phi}{\partial z} = 0 ~~~\text{dans}\{z = \zeta(t,x)\} 
\end{equation}

De manière analogue, on a aussi équivalence entre (\ref{eq_B3}) et 

\begin{equation} \label{eq_B3_bis}
    \nabla_\textbf{X}b\cdot\nabla_\textbf{X}\phi - \frac{\partial\phi}{\partial z} = 0 ~~~\text{dans}\{z = b(x)\} 
\end{equation}



\subsection{Formulation de Craig-Sulem-Zakharov}
\subsection{Adimensionnement des paramètres du système}

\begin{center}
    \texttt{ajouter schéma}
\end{center}

On va chercher à réécrire ces équations en faisant apparaître les rapports entre les différentes grandeurs caractéritiques qui entrent en jeux. Ainsi la négligeabilité d'un paramètre relativement à un autre s'impactera sur nos équation sous la forme d'une possible simplification de un ou plusieurs termes. Une équation ne faisant intervenir que des rapports adimmensionés est dites adimmensionelle. 
\\

Dans notre cas, les grandeurs caractéristiques de notre système sont

\begin{list}{\textbullet}{}
    \item $H_0$ est la profondeur du système.
    \item $a_{\text{surf}} = max(|\zeta|)$ est l'amplitude de la surface du fluide.
    \item $a_{\text{bot}} = max(|b|)$ est l'amplitude du fond du fluide.
    \item $L_1$ est la longueur d'onde de $\zeta$ dans la direction $\textbf{e}_1$.
    \item $L_2$ est la longueur d'onde de $\zeta$ dans la direction transversale $\textbf{e}_2$ (si $\textbf{X} = \mathbb{R}^2)$.
    
 \end{list}
\,
\\

On étudiera les rapports suivants.

\begin{list}{\textbullet}{}
    \item $\varepsilon = \frac{a_{\text{surf}}}{H_0}$ le coefficient de \textit{non-linéarité} du système.
    \item $\mu =\frac{H_0^2}{L_1^2}$ le coefficient de \textit{profondeur}.
    \item $\beta = \frac{a_{\text{bot}}}{H_0}$ est le coefficient de \textit{dénivellation} du fond.
    \item $\gamma = \frac{L_1}{L_2}$ est le coefficient de \textit{transversalité}.
    \item $\epsilon = \frac{ a_{\text{surf}}}{L_1}$ le coefficient d'\textit{amplitude}.
    \item $t_0 = \frac{L_1}{\sqrt{gH_0}}$ est l'échelle de temps caractéristique.
\end{list}

~\\
Pour adimensionner nos équations, on pose l'isomorphisme linéaire suivant:

  
\begin{equation}
\begin{aligned}
      \mathcal{I}:~~&\mathbb{R}\times\textbf{X}\times\mathbb{R} &\longrightarrow ~~~ &\textbf{X}\times \mathbb{R} & \\
      &(t,(x_1,x_2),z) &\longmapsto &\left(\frac{t}{t_0},\left(\frac{x_1}{L_1},\frac{x_2}{L_2} \right),\frac{z}{H_0}\right) ~~~&\text{si }\textbf{X}=\mathbb{R}^2\\
      &(t, x ,z) &\longmapsto &\left(\frac{t}{t_0},\frac{x}{L_1},\frac{z}{H_0}\right) ~~~&\text{si }   \textbf{X}=\mathbb{R}
\end{aligned}
\end{equation}

    
   
    


~\\

On effectue le changement de variable suivant.

\begin{list}{\textbullet}{}
    \item $(t',x',z') = \mathcal{I} (t,x,z)$
    \item $\zeta' = \dfrac{1}{a_{\text{surf}}}(\zeta \circ \mathcal{I}^{-1})$ 
    \item $b' = \dfrac{1}{a_{\text{bot}}}(b\circ\mathcal{I}^{-1})$
     \item $\phi' = \dfrac{1}{g t_0 a_{\text{surf}}}(\phi \circ \mathcal{I}^{-1})$
     \item $P' = \dfrac{t_0^2}{H_0^2\rho_0} (P \circ \mathcal{I}^{-1})~~~~~~~$   (et $P_{\text{atm}}' = \dfrac{t_0^2}{H_0\rho_0} P_\text{atm}$)
\end{list}

Ceci nous donne, entre autre, 
\begin{align}
    &-1 \leq \zeta' \leq 1\\ &-1 \leq b' \leq 1 
\end{align}
\subsubsection{Adimensionnement des équations de Bernoulli pour $\textbf{X} = \mathbb{R}$}
Pour $\textbf{X} = \mathbb{R}$, l'adimensionnement de (\ref{eq_B1}) s'éffectue ainsi:
\begin{align}
    \text{(\ref{eq_B1})} &\Leftrightarrow \frac{g t_0 a_{\text{surf}}}{t_0}\partial_{t'} \phi' + \frac{1}{2}\left(\frac{(g t_0 a_{\text{surf}})^2}{L_1^2}(\partial_{x'}\phi')^2+ \frac{(g t_0 a_{\text{surf}})^2}{H_0^2}(\partial_{z'}\phi')^2\right) + g H_0 z' = \frac{1}{\rho_0}\frac{H_0^2\rho_0}{t_0^2}(P'-P'_{\text{atm}}) \nonumber\\
    &\text{On factorise le terme non linéaire par $ \frac{(g t_0 a_{\text{surf}})^2 }{L_1^2}$ , ce qui fait apparaître $\mu$.}\nonumber\\
     &\Leftrightarrow  g  a_{\text{surf}} \partial_{t'} \phi' + (g t_0 a_{\text{surf}})^2\frac{1}{2L_1^2}\left( (\partial_{x'}\phi')^2+ \frac{1}{\mu}(\partial_{z'}\phi')^2\right) + g H_0 z' =  \frac{H_0^2  }{t_0^2}(P'-P'_{\text{atm}})\nonumber\\
     &\text{On développe $gt_0$ dans ce terme.}\nonumber\\
     &\Leftrightarrow  g  a_{\text{surf}} \partial_{t'} \phi' + 
     ( \sqrt{g}\frac{L_1}{\sqrt{H_0}} a_{\text{surf}})^2
     \frac{1}{2L_1^2}
     \left( (\partial_{x'}\phi')^2+ \frac{1}{\mu}(\partial_{z'}\phi')^2\right) 
     + g H_0 z' 
     = \frac{H_0^2 }{t_0^2}(P'-P'_{\text{atm}})\nonumber\\
     &\text{On simplifie ce terme.}\nonumber\\
    &\Leftrightarrow  g  a_{\text{surf}} \partial_{t'} \phi' + 
     \frac{g a_{\text{surf}}^2}{2H_0}
     \left( (\partial_{x'}\phi')^2+ \frac{1}{\mu}(\partial_{z'}\phi')^2\right) 
     + g H_0 z' 
     = \frac{H_0^2  }{t_0^2}(P'-P'_{\text{atm}})\nonumber\\
     &\text{On voit apparaître $\varepsilon$.}\nonumber\\
     &\Leftrightarrow  g  a_{\text{surf}} \partial_{t'} \phi' + 
     \varepsilon\frac{g a_{\text{surf}}}{2}
     \left( (\partial_{x'}\phi')^2+ \frac{1}{\mu}(\partial_{z'}\phi')^2\right) 
     + g H_0 z' 
     = \frac{H_0^2 }{t_0^2}(P'-P'_{\text{atm}})\nonumber\\
     &\text{On divise tout par $g  a_{\text{surf}}$ et on voit à nouveau apparaître $\varepsilon$ et $\mu$}. \nonumber\\
     &\Leftrightarrow   \partial_{t'} \phi' + 
     \frac{\varepsilon}{2}
     \left( (\partial_{x'}\phi')^2+ \frac{1}{\mu}(\partial_{z'}\phi')^2\right) 
     + \frac{z'}{\varepsilon} 
     = \frac{\mu}{\varepsilon}(P'-P'_{\text{atm}}) \label{eq_B1_adim}
\end{align}
\\

L'adimensionnement de (\ref{eq_B2}) s'effectue ainsi:
\begin{align}
    \text{(\ref{eq_B2})} &\Leftrightarrow  \frac{(g t_0 a_{\text{surf}})}{L_1^2}\partial_{x' x'}\phi'+ \frac{(g t_0 a_{\text{surf}})}{H_0^2}\partial_{z' z'}\phi'  = 0\nonumber\\
    &\text{On divise le tout par $ \frac{(g t_0 a_{\text{surf}})^2 }{H_0^2}$ , ce qui fait apparaître $\mu$.}\nonumber\\
    &\Leftrightarrow \partial_{z'z'}\phi' + \mu \partial_{x_1' x_1'}\phi'  = 0 \label{eq_B2_adim}
\end{align}

L'adimensionnement de (\ref{eq_B3}) s'effectue ainsi

\begin{align}
\text{(\ref{eq_B3})} &\Leftrightarrow \text{(\ref{eq_B3_bis})}& ~ \nonumber\\
&\Leftrightarrow \partial_x b \partial_x\phi - \partial_z\phi = 0 & \text{dans }\{z = b(x) - H_0\} \nonumber\\
&\Leftrightarrow \left(\frac{a_{\text{bot}}}{L_1} \partial_{x'} b'\right)\left(\frac{gt_0a_{\text{surf}}}{L_1}\partial_{x'}\phi'\right)  - \frac{gt_0a_{\text{surf}}}{H_0}\partial_{z'}\phi' = 0& \text{dans }\{z' = \beta b'(x') - 1\}\nonumber\\
&\Leftrightarrow  \frac{a_{\text{bot}}H_0}{L_1^2} \partial_{x'} b'  \partial_{x'}\phi'   - \partial_{z'}\phi' = 0& \text{dans }\{z' = \beta b'(x') - 1\}\nonumber\\
&\Leftrightarrow  \beta \mu \partial_{x'} b'  \partial_{x'}\phi'   - \partial_{z'}\phi' = 0& \text{dans }\{z' = \beta b'(x') - 1\}\label{eq_B3_adim}
\end{align}

L'adimensionnement de (\ref{eq_B4}) s'effectue ainsi

\begin{align}
\text{(\ref{eq_B4})} &\Leftrightarrow \text{(\ref{eq_B4_bis})}& ~ \nonumber\\
&\Leftrightarrow \partial_t\zeta + \partial_x \zeta \partial_x\phi - \partial_z\phi = 0 & \text{dans }\{z = \zeta(x,t)\} \nonumber\\
&\Leftrightarrow \frac{a_{\text{surf}}}{t_0}\partial_{t'} \zeta' + \left(\frac{a_{\text{surf}}}{L_1} \partial_{x'} \zeta '\right)\left(\frac{gt_0a_{\text{surf}}}{L_1}\partial_{x'}\phi'\right)  - \frac{gt_0a_{\text{surf}}}{H_0}\partial_{z'}\phi' = 0& \text{dans }\{z' = \varepsilon \zeta(t',x')\}\nonumber\\
&\Leftrightarrow \partial_{t'}\zeta' +  \frac{gt_0^2a_{\text{surf}}}{L_1^2} \partial_{x'} \zeta ' \partial_{x'}\phi'  - \frac{gt_0^2}{H_0}\partial_{z'}\phi' = 0&  \text{dans }\{z' = \varepsilon \zeta(t',x')\} \nonumber \\
&\Leftrightarrow \partial_{t'}\zeta' +  \varepsilon \partial_{x'} \zeta ' \partial_{x'}\phi'  - \frac{1}{\mu}\partial_{z'}\phi' = 0&  \text{dans }\{z' = \varepsilon \zeta(t',x')\} \label{eq_B4_adim}
\end{align}






Par la suite, lorsqu'on travaillera sur les équations adimensionnées, on se permettra d'omettre les apostrophes sur ces variables.
\newpage
\section{Cas du fond plat: Les équations de Korteweg-de Vries} 
\subsection{Obtention des équations de KdV à partir des équations d'Euler}
Dans cette partie on s'intéressera au cas où $\textbf{X} = \mathbb{R}$ et $b = 0$. D'après, (\ref{eq_B1_adim}), (\ref{eq_B2_adim}), (\ref{eq_B3_adim}) et (\ref{eq_B4_adim}), on se retrouve alors à étudier le système suivant.

\begin{align}
~&\partial_{t} \phi + 
     \frac{\varepsilon}{2}(\partial_{x}\phi)^2+ \frac{\varepsilon}{2\mu}(\partial_{z}\phi)^2
     + \frac{z}{\varepsilon} 
     = \frac{\mu}{\varepsilon}(P-P_{\text{atm}}) &\text{ si } -1 < z < \varepsilon \zeta(t,x)\\
~&\partial_{zz}\phi + \mu \partial_{x x}\phi  = 0  &\text{ si } -1 < z < \varepsilon \zeta(t,x) \label{eq_k2}\\
~&\partial_{z}\phi = 0&\text{ si } z  = - 1\label{eq_k3}\\
~&\partial_{t}\zeta +  \varepsilon \partial_{x} \zeta  \partial_{x}\phi  - \frac{1}{\mu}\partial_{z}\phi = 0 &\text{ si } z = \varepsilon \zeta(t,x)
\end{align}

Plus précisément, on s'intéressera au cas où $\mu << 1$.
\begin{prop}
    Si $\phi$ est de classe $2n+2$, alors il existe une fonction $f:\mathbb{R}^+\times\textbf{X}\rightarrow \mathbb{R}$, de classe $2n+2$, telle que
    \begin{equation}
        \phi (t,x,z) = \sum_{j = 0}^{n} \mu^j\frac{(-1)^j}{(2j)!} (z+1)^{2j}\frac{\partial^{2j}f}{\partial x^{2j}}(x,t)
        + (-\mu)^{n+1}\int_{-1}^z\frac{(z-s)^{2n+2}}{(2n+2)!}\frac{\partial^{2n+2} \phi}{\partial x^{2n+2}}(t,x,s)ds
    \end{equation}
\end{prop}
\begin{proof}
Le développement de Taylor de $\phi$ en $z = -1$, à l'ordre $2n+2$, (avec reste intégrale).
\begin{equation*}
\begin{split}
    \phi(t,x,z) = &\phi(t,x,-1) + (z+1)\frac{\partial \phi}{\partial z}(t,x,-1)\\
    &+ \frac{(z+1)^2}{2}\frac{\partial^2 \phi}{\partial z^2}(t,x,-1) + \frac{(z+1)^3}{3!}\frac{\partial^3 \phi}{\partial z^3}(t,x,-1)\\
    & \cdots\\
    &+ \frac{(z+1)^{2n}}{(2n)!}\frac{\partial^{2n} \phi}{\partial z^{2n}}(t,x,-1) + \frac{(z+1)^{2n+1}}{(2n+1)!}\frac{\partial^{2n+1} \phi}{\partial z^{2n+1}}(t,x,-1)\\
    &+\int_{-1}^z\frac{(z-s)^{2n+2}}{(2n+2)!}\frac{\partial^{2n+2} \phi}{\partial z^{2n+2}}(t,x,s)ds
\end{split}
\end{equation*}

L'équation (\ref{eq_k2}) nous donne, après une récurrence immédiate, pour $z>-1$.
\begin{equation*}
    \frac{\partial^{2j} \phi}{\partial z^{2j}}(t,x,z) = (-\mu)^{n}\frac{\partial^{2j} \phi}{\partial x^{2j}}(t,x,z)
\end{equation*}

\begin{equation*}
    \frac{\partial^{2j+1} \phi}{\partial z^{2j+1}}(t,x,z) = (-\mu)^{n}\frac{\partial^{2j}}{\partial x^{2j}}\frac{\partial \phi}{\partial z}(t,x,z)
\end{equation*}

En faisant tendre $z$ vers -1, il vient que les dérivées partielles $\frac{\partial^{k} \phi}{\partial z^{k}}(t,x,z)$ sont bien définies en $z = -1$ et 
\begin{equation*}
    \frac{\partial^{2j} \phi}{\partial z^{2j}}(t,x,-1) = (-\mu)^{n}\frac{\partial^{2j} \phi}{\partial x^{2j}}(t,x,-1)
\end{equation*}
\begin{equation*}
    \frac{\partial^{2j+1} \phi}{\partial z^{2j+1}}(t,x,-1) = (-\mu)^{n}\frac{\partial^{2j}}{\partial x^{2j}}\frac{\partial \phi}{\partial z}(t,x,-1)
\end{equation*}

De plus, l'équation (\ref{eq_k3}), nous donne $\frac{\partial \phi}{\partial z}(t,x,-1) = 0$ et donc 
\begin{equation*}
    \frac{\partial^{2j+1} \phi}{\partial z^{2j+1}}(t,x,-1) = 0
\end{equation*}

Finalement, on obtient

    \begin{equation*}
        \phi (t,x,z) = \sum_{j = 0}^{n} \mu^j\frac{(-1)^j}{(2j)!} (z+1)^{2j}\frac{\partial^{2j}\phi}{\partial x^{2j}}(t,x,-1)
        + (-\mu)^{n+1}\int_{-1}^z\frac{(z-s)^{2n+2}}{(2n+2)!}\frac{\partial^{2n+2} \phi}{\partial x^{2n+2}}(t,x,s)ds
    \end{equation*}


Ce qui, si on pose $f(t,x) = \phi(t,x,-1)$, nous donne le résultat.



\end{proof}





\end{document}
\paragraph{Exemple :} Après ces changements de variables, (\ref{eq_B2}) devient $$\partial_{z'z'}\phi' + \mu \partial_{x_1' x_1'}\phi'  = 0 ~~~ \text{si }\textbf{X} = \mathbb{R} $$
$$\partial_{z'z'}\phi' + \mu \partial_{x_1'x_1'}\phi' + \mu \gamma^2 \partial_{x_2'x_2'}\phi' = 0 ~~~ \text{si }\textbf{X} = \mathbb{R}^2$$